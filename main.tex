\documentclass[
  12pt,
  paper=a4,
]{scrartcl} % ->
% -> KOMA.
%    http://mirrors.ctan.org/macros/latex/contrib/koma-script/doc/scrguien.pdf

\newcommand{\thisdocversion}{\detokenize{2021_10_20_0}}


% Font support (Xelatex)
\usepackage{fontspec}

% Langeuage support (replaces babel)
\usepackage{polyglossia}
\setmainlanguage{english}

% Basic math support
\usepackage[fleqn]{mathtools} % Fixes amsmath + other goodness
\usepackage{amsmath,amssymb,amsthm}

\mathtoolsset{centercolon}


% Multiline comment (\begin{comment}...\end{comment})
\usepackage{comment}


% Colors
\usepackage[svgnames,dvipsnames]{xcolor}
% xcolor -> usenames option is obsolete

% Nice lists
\usepackage{enumerate}
\usepackage{enumitem}

% Things To-Do. -> \todo,  \missingfigure
%\usepackage[textsize=scriptsize,shadow]{todonotes}

% finicky micro-typography
\usepackage{microtype}

% Underline support -> \ul
\usepackage[normalem]{ulem}


% Lorem ipsum -> \blindtext
\usepackage{blindtext}
% Alternative:
% \usepackage{lipsum}

% -> \stackrel
\usepackage{stackrel}

% Links and cross-refrences
\usepackage{hyperref}
\usepackage[nameinlink]{cleveref}

\hypersetup{
    colorlinks=true,
    urlcolor=blue,
}


% Source code blocks with highlighting
% TODO: test if tectonic and if so enable 
% DOC: https://ctan.javinator9889.com/macros/latex/contrib/minted/minted.pdf
%\usepackage[cache=false]{minted}
\usepackage[outputdir=./tmp/]{minted}
% https://tex.stackexchange.com/questions/343494/
\usepackage{xpatch}
\makeatletter
\AtBeginEnvironment{minted}{\dontdofcolorbox}
\def\dontdofcolorbox{\renewcommand\fcolorbox[4][]{##4}}
\xpatchcmd{\inputminted}{\minted@fvset}{\minted@fvset\dontdofcolorbox}{}{}
\xpatchcmd{\mintinline}{\minted@fvset}{\minted@fvset\dontdofcolorbox}{}{} % see https://tex.stackexchange.com/a/401250/
\makeatother



% Bibliography
% ... not worth the trouble for notes ...
%  but watch: https://github.com/tectonic-typesetting/tectonic/issues/35


% Symbols
% See: https://www.math.utk.edu/~finotti/sm14/m504/symbols.pdf
% Only include some symbols
\usepackage[only,varoplus,varotimes]{stmaryrd}
\usepackage{wasysym}
% --> \mathbbm{1} (characteristic function)
\usepackage{bbm}
\usepackage{bm}

% Bra-ket notation
\usepackage{braket}

%\pdfminorversion=7

\usepackage{nicematrix}
\usepackage{tikz}

%%%%%%%%%%%%%%%%%%%
%%% PAGE LAYOUT %%%
%%%%%%%%%%%%%%%%%%%

% \newcommand{\mysep}{\vspace{.2in} \hrule \vspace{-.1in}}


% Horizontal rule that disappears if at the top of the page
\newcommand\disappearingrule{%
  \par % make sure we end a paragraph
  \leaders\vrule width \textwidth\vskip0.6pt % rule thickness
  \nointerlineskip % disable interline glue here
  \vskip10pt % space below the rule
}

% Add a rule before the section titles.
\makeatletter
    \renewcommand{\sectionlinesformat}[4]{%
        \Ifstr{#1}{section}{
            \disappearingrule
            \textbf{}\@hangfrom{\hskip #2#3}{#4}%
        }{%else (not a section; use default definition)
            \textbf{}\@hangfrom{\hskip #2#3}{#4}%
        }
    }
\makeatother

\usepackage[open,openlevel=1]{bookmark}
\bookmarksetup{}


% KOMA-script options [should be after setting fonts, etc.]
% TODO: try margins etc with \areaset (KOMA)
\KOMAoptions{
  % page layout using DIV-part division (do NOT use geometry except for showframe)
  % with selected font: 11 -> 80 ch/line, 12 -> 90 ch/line, 13 -> 100 ch/line
  DIV=13,
  %twoside=semi, % ->
  % -> two-sided printing with one-sided margins and one-sided marginal notes
  numbers=noenddot % Sections without a dot
}

% KOMA page options
% \pagestyle{myheadings}
\usepackage{scrlayer-scrpage}
\renewcommand{\headfont}{\normalfont} % No italics
% \clearpairofpagestyles % Clear all defaults
\setkomafont{author}{\titlefont\normalfont}
\setkomafont{date}{\titlefont\normalfont}
\KOMAoptions{headsepline = true, footsepline = true}
\makeatletter
% \ihead{\protect\@title}
% \ohead{\protect\studentname}
\ohead{\protect\@title}
\cfoot{\pagemark}
\makeatother



% TODO: redo this more elegantly...
\setlength{\parindent}{0pt}
\setlength{\parskip}{3pt}


\newcommand{\norm}[1]{\|#1\|}
\newcommand{\normtwo}[1]{\|#1\|_2}
\newcommand{\normtwolr}[1]{\left\|#1\right\|_2}
\newcommand{\normone}[1]{\|#1\|_1}
\newcommand{\norminf}[1]{\|#1\|_\infty}
\newcommand{\normfrob}[1]{\|#1\|_{\text{F}}}
\newcommand{\transp}{^{\mathrm{T}}}

\newcommand{\RR}{\mathbb R} % Real numbers
\newcommand{\CC}{\mathbb C} % Complex numbers
\newcommand{\Kry}{\mathcal{K}} % Krylov subspace
\newcommand{\OO}{\mathcal{O}} % Big-O
\newcommand{\PP}{\mathcal{P}} % polynomials of degree <= n
\newcommand{\rP}{\mathrm{P}} % polynomials with p(0)=1

\DeclareMathOperator{\Ker}{Ker} % Kernel
\DeclareMathOperator{\rank}{Rank}
\DeclareMathOperator{\range}{Range}
\DeclareMathOperator{\sspan}{Span}
\DeclareMathOperator{\trace}{trace}
\DeclareMathOperator{\diag}{diag}


\makeatletter
\newcommand{\hathat}[1]{% 
\begingroup%
  \let\macc@kerna\z@%
  \let\macc@kernb\z@%
  \let\macc@nucleus\@empty%
  \hat{\raisebox{.2ex}{\vphantom{\ensuremath{#1}}}\smash{\hat{#1}}}%
\endgroup%
}
\makeatother



\newcommand{\studentname}{Juan Antonio Pedreira Martos}
\title{NUMA11/FMNN01 -- Notes}
\author{\studentname}
% \date{September 5, 2021}
\date{\footnotesize{Version \texttt{\thisdocversion}}}
% Monkey patching the vertical space...
\addtokomafont{author}{\vspace*{-0.7em}}
\addtokomafont{date}{\vspace*{-0.3em}}
\addtokomafont{title}{\vspace*{-0.5em}} % Affects all
\addtokomafont{titlehead}{\vspace*{-6em}}



\begin{document}

% \makeatletter
% \texttt{\meaning}
% \makeatother

\maketitle
\vspace*{-2.5em}
% \vspace*{-0.5em}

These are my notes for the course ``NUMA11/FMNN01 Numerical Linear Algebra'' taught at Lund University in Autumn 2021 by the professors Claus Führer and Philipp Birken.

\addsec{Preliminaries}

Standard disclaimer: all mistakes/typos are mine. I am not responsible if this document causes any problem and/or scares your cat. Some parts are intentionally not mathematically rigorous for the sake of space and clarity.

I appreciate any constructive criticism/corrections/suggestions.

Conventions: Vectors (1D) are treated as single-column matrices (2D) by default and are always typeset in boldface: $\bm v$. To denote a row vector, use transpose ($\transp$) or Hermitian transpose ($^*$) of a vector.

Acronyms: RHS/LHS means Righ/Left-hand Side of an equation. ``iff'' means ``if and only if''.

\addsec{References}

\begin{itemize}
    \item {} [Lecture slides here].
    
    \item Trefethen, L.\@ N., Bau, D.\@ (1997). \emph{Numerical Linear Algebra}. 1st edition. SIAM. ISBN: 9780898713619.
    \item Golub, G.\@ H., van Loan, C.\@ F.\@ (2013). \emph{Matrix Computations}. 4th edition. JHU~Press. ISBN: 9781421407944.
    \item {} [TODO: finish list of references / use bibtex or biblatex]
\end{itemize}

\newpage
\addsec{Basic Linear Algebra notions}
Assumed known concepts: vector space, scalar, vector, dimension, linear combination, span of vectors, basis, collinear vectors, linear in/dependance, vector subspace.%, positive / negative (semi)definite.

\begin{description}
\item[Matrix:] Sequence of numbers laid out in a ``table'' $A$ of size ${m\times n}$, that is, $m$ rows (vertical vectors, ordered top-to-bottom), by $n$ columns (vertical vectors, ordered left-to-right); $m\cdot n$ entries in total.

These numbers (scalars) can be real $\RR$ or complex $\CC$ (or in any field in general...).

Individual entries are denoted by $a_{ij}$ or (to avoid confusion) $(A)_{i,j}$.

Columns of a matrix $A$ are denoted $\bm a_i$ and rows are denoted $ \bm a_{i,:}\transp \in \RR^{1\times n}$.

The entries $a_{ii}$ ($1 \le i\le \min(m,n)$) form the diagonal of the matrix $A$.

\item[Linear mapping:] a.k.a.\@ linear operator, vector space hom(e)omorphism.

Function $f$ between real (or complex) vector spaces \[\quad f: \RR^m \to \RR^n ;\quad \bm x \mapsto f(\bm x) = \bm y; \quad \bm x \in \RR^m, \, \bm y \in \RR^n .\]
Must meet linearity requirements:
\begin{itemize}
    \item Mapping of the sum is the sum of the mappings: $\bm a + \bm b \mapsto f(\bm a) + f(\bm b)$.
    \item Scaling of the input maps to the scaling of output (homogeneity): $\alpha\bm a \mapsto \alpha f(\bm a)$; this includes the case $\bm 0\mapsto \bm 0$.
\end{itemize}

[Pedantic remark: the input or output spaces of a real linear mapping $f$ may not actually be $\RR^m$ or $\RR^n$, but at least they must be \emph{isomorphic} to them; also, we always assume finite-dimensional vector spaces.]

Any output vector $\bm y$ is a linear combination of $n$ vectors (in $\RR^m$) called \emph{columns}; the $n$ coefficients of the linear combination are directly given by the coordinates of the input vector in a given basis.

If we select fixed bases for both the input and output spaces, then this operator is uniquely determined by a matrix that relates the coordinates between spaces and vice~versa: \[
    A \in\RR^{m\times n} \leftrightarrow  A\bm x = f(\bm x) = \bm y
\]
In this case (i.e.\@ given the operator and bases), we can talk about the ``matrix operator'' and ``action of the matrix'' on a vector. And we use then the term ``matrix'' loosely to refer both to the ``linear operator'' and the associated ``table of numbers'' in a given basis. Also the \emph{columns} correspond to the columns of the matrix, of course.

We distinguish between \emph{basis}-independant/-invariant properties of a mapping/matrix (e.g.\@ rank, eigenvalues, etc.) and  \emph{basis}-dependant ones (e.g. entries of $A$, eigenvectors).

Usually, unless otherwise stated, everything is implicitly referred with respect to the ``canonical'' basis (vectors $\bm e_i = (..., 0 , 1, 0, ...)$). The list of coordinates of any vector in the canonical basis is considered the ``absolute'' representation of a that vector (tuple or list of numbers, called ``components'' of the vector).

Moreover, when $m=n$ we usually assume that the input basis is the same as the output basis.

IMPORTANT: the same linear operation on vectors can be described by different matrices (i.e.\@ different ``tables of numbers'') if the basis is changed (typical example: diagonalization).

\item[Square matrix:] Matrix where $m=n$; for any associated mapping, the input and output spaces are the same; the mapping is called \emph{endomorphism}.

$n$ is sometimes called the \emph{order} of the matrix $A$.

Matrices that are not square are called rectangular.

\item[Diagonal matrix:] Matrix where all entries outside the diagonal are zero. (Not necessarily square). Note that this property \emph{does} depend on the basis.

\item[Identity matrix:] Square diagonal matrix $I$, with all ones in the diagonal. When the input and output bases are the same, the associated operator is the identity (input and output vectors are the same).

\item[Zero matrix:] Matrix with all zeros $0\in\RR^{m\times n}$. Zero operator (the output space is the zero vector), absorbing element (multiplying by a zero matrix always results in a zero matrix, of the suitable size), basis-independent property.

\item[Transpose of a matrix:] (a.k.a.\@ transposed matrix) Matrix $A\transp$ that results by exchanging the rows and columns of the ``table'' (like a 2D symmetry with respect to the diagonal).
\[
A\in\RR^{m\times n} \implies A\transp\in \RR^{n\times m},\,\,
(A\transp)_{ij} = a_{ji}
\]
For any associated mapping, the input and output spaces are also exchanged as a consequence.

\item[Symmetric matrix:] Matrix $A$ such that $A = A\transp$ (necessarily square).

Main idea: the action performed by the (linear combination of the) columns is the same as the action performed by the rows.

\item[Hermitian transposed matrix:] matrix $A^* = \mathrm{conj}(A\transp)$; it is the matrix  the results from transposition and complex conjugation of $A$.

If $A$ is real, then $A\transp = A^{*}$.

Many results on real symmetric matrices can be generalized by just replacing $A\transp$ with $A^*$.

\item[Hermitian matrix:] Matrix $A$ such that $A = A^*$ (necessarily square).

If $A$ is real symmetric then it is also Hermitian.

Property: the diagonal of $A$ Hermitian must be real.

\item[Inner/dot product:]
$\displaystyle\bm x^* \bm y =\sum_{i=1}^n \bar x_i^*\, y_i$; if vectors are real: $\displaystyle\bm x\transp \bm y =\sum_{i=1}^n x_i\, y_i$.

Note: this is the ``standard'' inner product (with Gram matrix $G=I$).

\item[Euclidean norm:] (induced by the standard inner product) \[\|\bm x\| = \normtwo{\bm x}= \sqrt{\bm x^* \bm x} = \sqrt{\sum_i |x_i|^2}.\]

This is the norm used ``by-default'' for vectors. It the same as the 2-norm ($p$-norm with $p=2$) (more info in the norm section).

\item[Angle between two vectors:]{} Geometric angle $\alpha_{\bm x,\bm y}$,  between nonzero vector pairs $\bm x,\bm y$.

Defining property: 
\[\bm x^* \bm y = \normtwo{\bm x}\,\normtwo{\bm y}\,\cos \alpha_{\bm x,\bm y}\]

\item[Unit/Normalized vector:] $\bm x$ is unit vector when $\|\bm x\|=1$.

\item[Matrix-by-vector product:] Application of the matrix operator $A$ on a given vector:
\[
A\in\RR^{m\times n},\,\bm x\in \RR^n;\quad A\bm x \in \RR^m
\]

Can be expressed both as linear combination of columns and as dot products with the rows:
\begin{align*}
A\bm x &= 
    (\bm a_1| \cdots | \bm a_n)\, \bm x=
     \bm a_1 x_1 + \cdots + \bm a_n x_n=
    \sum_{j=1}^n \bm a_j x_n
\\
A\bm x &=
\begin{pNiceMatrix}[hlines]
        \bm a_{1,:}\transp
    \\
        \vdots 
    \\
        \bm a_{m,:}\transp
\end{pNiceMatrix}
\, \bm x
=
\begin{pNiceMatrix}[hlines]
        \bm a_{1,:}\transp \bm x
    \\
        \vdots
    \\
        \bm a_{m,:}\transp \bm x
\end{pNiceMatrix}
\end{align*}

Each entry is the result of a dot product with a row:
\[
(A\bm x)_i =
    \bm a_{i,:}\transp,\bm x
    =
    \sum_{k=1}^n a_{ik}\, x_k,
    \quad
    i \in \{1,..., m\}
\]

\item[Matrix-by-matrix product:] Composition of two matrix operations in order.

Intermediate output and input spaces have to match, meaning that: the product $AB$ is defined for $A\in\RR^{m\times n}$, $B\in\RR^{p\times q}$ only when $n=p$; then, the shape of the product matrix $AB$ is $m\times q$.

As a function composition, it has the associative property: $(AB)C = A(BC) = ABC$; parentheses are not needed, because grouping order is not ambiguous.

It can be expressed in terms of matrix-by-vector operations:
\begin{align*}
A B = 
\begin{pNiceMatrix}[vlines]
    A\, \bm b_1
    &  \cdots &
    A\, \bm b_q
\end{pNiceMatrix}
&=
\begin{pNiceMatrix}[vlines]
        (\bm a_1| ... | \bm a_n)\, \bm b_1
    & \cdots &
        (\bm a_1| ... | \bm a_n)\, \bm b_q
\end{pNiceMatrix}
\\[10pt]
&=
\begin{pNiceMatrix}[vlines]
    \sum_{k=1}^n \bm a_k b_{k1}
    & \cdots &
    \sum_{k=1}^n \bm a_k b_{kq}
\end{pNiceMatrix}
\end{align*}
\begin{align*}
A B =
\begin{pNiceMatrix}[hlines]
    \bm a_{1:}\transp
    \\ \vdots \\
    \bm a_{m:}\transp
\end{pNiceMatrix}
\begin{pNiceMatrix}[vlines]
    \bm b_{1}
    & \cdots &
    \bm b_{q}
\end{pNiceMatrix}
&=
\begin{pNiceMatrix}[hlines,vlines]
        \bm a_{1,:}\transp, \bm b_{1}
    & \cdots &
        \bm a_{1,:}\transp, \bm b_{q}
\\
    \vdots & \ddots & \vdots
\\
        \bm a_{m,:}\transp, \bm b_{q}
    & \cdots &
        \bm a_{m,:}\transp, \bm b_{q}
\end{pNiceMatrix}
\end{align*}

Each matrix entry can be expressed as a row-by-column dot product:
\[
    (AB)_{ij}= \bm a_{i,:}\transp, \bm b_j =
    \sum_{k=1}^n a_{i,k} \, b_{k,j}
\]

Finally, $AB$ can be also expressed as a sum of column-by-row outer products:
\[
AB =
\begin{pNiceMatrix}[vlines]
    \bm a_{1}
    & \cdots &
    \bm a_{n}
\end{pNiceMatrix}
\begin{pNiceMatrix}[hlines]
    \bm b_{1:}\transp
    \\ \vdots \\
    \bm b_{n:}\transp
\end{pNiceMatrix}
=
\sum_{k=1}^{n} \bm a_k \, \bm b_{k:}\transp
\]

[TODO: I don't like the notation $\bm a_{i,:}\transp$, find something better...]

[TODO: alternative approach using Einstein notation (by understanding this notation, all previous interpretations become obvious).

See: \url{https://ajcr.net/Basic-guide-to-einsum/}
]

Properties of transposed of a product:
\begin{itemize}
    \item $(AB)\transp= B\transp A\transp$
    \item $(A_1A_2...A_{N-1}A_N)\transp =
            A_N\transp A_{N-1}\transp...A_2 \transp A_1 \transp$
    \item same with $(^*)$ instead of $(\transp)$
\end{itemize}

\item[Permutation matrix:] Square matrix with exactly one $1$ per column (hence, per row), the rest of entries are zeros. It is a ``permutation'' of the identity matrix.

It is used to permute (reorder) the rows or columns of a target matrix $A$:
\begin{itemize}
    \item Pre-multiplication permutes the rows of $A$: $PA$.
    \item Post-multiplication permutes the columns of $A$: $AP$.
\end{itemize}

Property: permutations are orthogonal ($P^{-1}=P\transp$).


\item[Range of a matrix:] $\range(A)$ is the vector space generated by the span of the columns of $A^{m\times n}$. It is a subspace of $\RR^n$. It does not depend on the basis (but the resulting coordinates are dependent).

\item[Rank of a matrix:] Dimension of the range of a matrix, that is, the number of independent columns.

Property: $\rank(A)=\rank(A\transp)$ (row and column ranks are equal).

Property: $\rank(A) \le \min(m, n)$.

The number $\rank(A) - \min(m, n)$ is called \emph{rank deficiency}.

\item[Full rank matrix:] Matrix with the highest possible rank, i.e.\@ $\min(m,n)$; the rank deficiency is zero. It is called full rank by columns if all columns are independent (i.e.\@ rank $n$, only happens with ``tall'' matrices $m\ge n$), and the same by rows (i.e rank $m$, ``fat'' matrices $m\le n$).

If $A$ is square ($m=n$): it is full rank by columns iff it is full rank by rows.

\item[Null space of a linear mapping/matrix:] (a.k.a.\@ kernel).

[...

\phantom{ }    TODO: explain this
    
...]

\item[Nullity of a matrix:]

Defined as: $\mathrm{Nullity}(A) = \dim(\Ker(A))$.

Note that the \emph{rank deficiency} is a different concept; but that they result the same when $m\ge n$ (including square matrices).


\item[Rank–nullity theorem:] 
\[
\rank(A) + \mathrm{Nullity}(A)
= n
\]
($n$ is the number of columns, i.e.\@ the dimension of the input space).

This is a dimension decomposition:
\[
\dim(\range(A)) + \dim(\Ker(A))
= n = \dim(\RR^n)
\]


\item[Inverse matrix:] $A^{-1}\in\RR^{n\times n}$ is the matrix associated with the inverse operator of $A$ (same size); that is, it allows to recover ''all`` possible inputs $\bm x$ from the output $\bm y=A\bm x$ by just applying it: $A^{-1}\bm y = A^{-1} A\bm x = \bm x$. $A A^{-1} = I$

$A^{-1}$ exists iff $A$ is full rank (then $A$ is called \emph{invertible}). If $A^{-1}$ exists it is unique. Invertibility of $A$ is a basis-invariant property.

Inverse of product property: $(AB)^{-1} = B^{-1}A^{-1}$, given $A$ and $B$ invertible.

\item[Singular matrix:] Square matrix that has no inverse (basis-invariant property).

The operator is non-bijective: for a given output $\bm y$ there may be zero (non-surjective) or multiple (non-injective) corresponding inputs $\bm x$.

An invertible matrix is also called regular.

\item[Property:] multiplication by invertible matrix preserves rank (and deficiency).

In particular: for $B$ invertible ($A$ might not be square)
\[\boxed{\rank(A) = \rank(AB) = \rank(BA)}\]
Proof:
\[
\bm x\in \Ker(A) \implies A\bm x=\bm 0 \implies B A \bm x = \bm 0 \implies \bm x\in \Ker(BA)
\]
Conversely, $B$ is invertible, then $\Ker(B) = \{\bm 0\}$.
\[ B\bm y = \bm 0 \implies \bm y = \bm 0 \]
Then:
\[\bm x \in \Ker(BA) \implies BA\bm x = \bm 0 \implies B\underbrace{(A\bm x)}_{\bm y} = \bm 0 \implies (A\bm x) = \bm 0 \implies \bm x \in \Ker(A) \]

[Shorter proof: $B A\bm x = \bm 0 \implies B^{-1}B A\bm x = B^{-1} \bm 0 \implies A\bm x = \bm 0$]

So, the deficiencies are the same and then $\rank(A) = \rank(BA)$.

Same argument with rows (premultiply by $\bm x\transp$): $\rank(A) = \rank(AB)$.

\item[Coordinates of a vector in a basis:] list of scalar coefficients used to (uniquely) represent a given vector as the linear combination of the vectors of a given basis.

\item[Change of basis:] change of representation as coordinates of a vector from a given basis into another.

An invertible (square) matrix uniquely represents a change of basis: if $A\bm x = \bm b$ with $A$ square has only one solution, we can interpret $\bm b$ as a vector (assume canonical basis for simplicity), and then $\bm x$ contains the coordinates in the basis defined by the columns of $A$.

\item[Similar matrices:] Two square matrices $A,\,B$ are called similar when there exists an invertible matrix $P$ such that $B= P^{-1}\, A\, P$.

Similarity defines an IMPORTANT equivalence relation: two similar matrices correspond to the same square mapping (endomorphism) under different bases, being $P$ the change of basis matrix. The pre- and post-multiplication correspond to the forward/inverse change of basis to/from the output/input spaces.

[Recall that we assume that for square matrices, the input and output spaces are referred to the same basis.]

When two square matrices are similar they share:
\begin{itemize}
    \item the rank
    \item the characteristic polynomial $\chi_A(\lambda)$
    \item the eigenvalues $\lambda_i$
    \item the eigenvectors (note though, that the coordinate representation changes as the basis is changed).
    \item the eigenvalue multiplicities: $\mu_a(\lambda_i)$ and $\mu_g(\lambda_i)$
    \item the trace and the determinant
    \item in general, all the properties that are basis-invariant
    \item if $A$ is Hermitian (/real symmetric) then $B$ is not in general also Hermitian, but this is guaranteed when $P$ is unitary (/orthogonal).
\end{itemize}

\item[System of linear equations:] $A\bm x = \bm b$, $A\in \RR^{m\times n}$. 

The system may have either: one solution, infinitely many solutions (they form a subspace) or no solutions (when $\bm b \notin \range(A)$).

These cases can be characterized by the ranks of the matrices $A$ and $(A|\bm b)$:
\begin{itemize}
    \item NO SOLUTIONS iff $\rank(A) \neq \rank(A|\bm b) = \rank(A) + 1$.
    
    The system is called incompatible.
    
    Reason: $\bm b$ does not lie in the range space of $A$ because it is not a linear combination of the columns of $A$, i.e.\@ $\bm b$ is linearly independent with them, so the resulting rank gets incremented by one.
    
    \item INFINITE SOLUTIONS iff $\rank(A) = \rank(A|\bm b) < n$  (where $n$ is the number of columns in $A$).
    
    Reason: $\bm b$ lies in the range space of $A$, and, at the same time, the columns of $A$ are linearly dependent, this implies that, for the vectors in the range space, there are multiple ways of being expressed as a linear combination of the columns of $A$. The vector $\bm b$ is therefore ambiguously representable.
    
    In particular, two solutions $\bm x_1$ and $\bm x_2$ with $A\bm x_1=A\bm x_2=\bm b$ are related by the nullspace of $A$ because: $A(\bm x_2 -\bm x_1) = A \bm x_\Delta = \bm 0$, so the set of all possible solutions is $\{\bm x_1 +  \bm x_\Delta : \, \forall \bm x_\Delta \in \Ker(A) \}$. 
    
    Then, there are $\dim(\Ker(A))=\mathrm{Nullity}(A)$ ``degrees of freedom''.
    
    \item Exactly ONE SOLUTION iff $\rank(A) = \rank(A|\bm b) = 
n$.
    
    Reason: there is exactly one linear combination of the columns of $A$ that results in $\bm b$; this is because, by the rank-nullity theorem, $\mathrm{Nullity}(A)=0$.
    
    If the matrix $A$ is square ($m=n$) it is invertible, and we can ensure in this case that the columns of $A$ form a basis of $\RR^n$ and thus all equations have one solution regardless of $\bm b$.

\end{itemize}

We can always convert a system with at least one solution...
\begin{itemize}
    \item ...into a system with NO solutions by adding a contradictory equation, e.g.\@ $0=1$ or copy one equation and change the RHS, etc.
    \item ...into a system with exactly the same solutions by adding a redundant equation, e.g.\@ $0=0$, repeating one of the previous equations, etc.
    \item ...into a system with infinite solutions (adds one more ``degree of freedom''), by adding a new unused variable ($A$ has a new column of zeros).
\end{itemize}

Regardless of the ranks and number of solutions:
\begin{itemize}
    \item if $m>n$, the system is called \emph{overdetermined} (used in least squares fitting).
    \item if $n>m$, it is called \emph{underdetermined}.
    \item and if $n=m$, it is called \emph{exactly determined}.
\end{itemize}

\item[Eigen-stuff:] an ``eigenvalue'' is a scalar $\lambda$ associated with a square linear operator $A$, such that there exists a nonzero vector $\bm v$, called eigenvector, with \[
    \boxed{A\bm v = \lambda\bm v}.
\] 
That is: input and output are collinear/parallel/proportional.

Note: all nonzero vectors collinear with $\bm v$ are eigenvectors (with the same $\lambda$).

Eigenvalues of $A$ are basis-invariant. There are as many eigenvalues as $n$ (matrix order), but some of them may be complex for real $A$ (prototypical example: rotations).

Each distinct eigenvalue $\lambda_i$ defines a subspace of associated eigenvectors. This set of vectors (+the zero vector) form a vector space called eigenspace (specifically, the $\lambda_i$-eigenspace). The dimension of each eigenspace is called geometric multiplicity $\mu_g(\lambda_i)$ (property of operator, basis-invariant).

We can consider two cases:
\begin{itemize}
    \item $\lambda_i\neq 0$, then, the action of $A$ does not change the direction of the input eigenvector $\bm v$.

    \item $\lambda_i=0$, then, the eigenspace is the nullspace; the input eigenvector is ``squished'' into the vector $\bm 0$. The $0$-eigenspace is precisely $\Ker(A)$.
\end{itemize}

Properties:
\begin{itemize}
    \item $A$ and $A\transp$ have same eigenvalues; the eigenvectors are different in general.
    \item $A$ is singular iff $\lambda=0$ is an eigenvalue. The eigenspace is $\Ker(A)$.
    \item $A$ and $A^{-1}$ (if exists) have inverse eigenvalues. The eigenvectors for each $\lambda_i$, $\lambda_i^{-1}$ are the same.
    \item $\alpha A$ has eigenvalues $\alpha\lambda_i$. The eigenvectors are the same as for $A$ and $\lambda_i$.
    \item $A^k$ has eigenvalues $\lambda_i^k$. Note that, for example, if $1$ and $-1$ are eigenvalues of $A$, they are merged into 1 for $A^2$. The eigenvectors are the same.
    \item If $p(x)$ is a polynomial, evaluating $p(A)$ results in a matrix that has eigenvalues $p(\lambda_i)$. The eigenvectors are the same.
    \item $(A-\mu I)$ has eigenvalues $(\lambda_i - \mu)$. The eigenvectors are the same.
    \item If $A=A^*$ (Hermitian or symmetric real), all eigenvalues are real. (Also, see later, they are orthogonally diagonalizable).
    \item If $A$ is Hermitian and (semi)-def.\@ positive/negative, all eigenvalues are (zero or) positive/negative.
    \item If $A$ has $n$ distinct eigenvalues, it is necessarily diagonalizable (because each eigenspace has at least dimension 1). 
    \item (Because of determinant property) The product of products of all eigenvalues of $A$ and all of $B$ is the same as the product of all eigenvalues of $AB$.
    \item $AB$ has eigenvalue $0$ iff $A$ or $B$ (or both) has eigenvalue $0$.
    \item In general, we cannot deduce the eigenvalues of $AB$ just from the eigenvalues of $A$ and $B$. Same for $A+B$.
\end{itemize}

\item[Trace of matrix:]{} $\trace(A)$ is the sum of the elements of the diagonal. If the matrix is square, it is equal to the sum of eigenvalues.

\item[Determinant:] weird operator $\det(\cdot)$ defined on square matrices (and linear mappings).

Difficult to compute (most efficient algorithms compute the eigenvalues first).

We mostly care only about it for the following properties:
\begin{itemize}
    \item it is uniquely defined for any square matrix
    \item it is basis-independent.
    \item it is zero iff the matrix is singular
    \item it is equal to the product of all eigenvalues
    \item the determinant of the matrix product is the product of determinants
    \item the determinant of the inverse is the inverse of the determinant
\end{itemize}

\item[Characteristic polynomial:] defined for a square matrix $A$, as \[\chi_A(\lambda) = {\det}(\lambda I - A),\] which is a polynomial on $\lambda$. $\lambda=\lambda_i$ is a root of $\chi_A(\lambda)$ iff it is eigenvalue of $A$. It is a basis-invariant property of the operator (endomorphism, as it is square).

Property: the constant term of [...] is  the minus  determinant of A. [TODO]

The multiplicity of a particular root/eigenvalue $\lambda_i$ in the polynomial $\chi_A(\lambda)$  is called arithmetic multiplicity $\mu_a(\lambda_i)$.

By the fundamental theorem of algebra: \[\sum_i \mu_a(\lambda_i) =n.\]
Also: $\mu_g(\lambda_i) \le \mu_a(\lambda_i)$.

Property: a square matrix is diagonalizable iff $\mu_g(\lambda_i) = \mu_a(\lambda_i)$.

Note: some books consider a real matrix non-diagonalizable when it has complex eigenvalues (those eigenvalues are not considered ``valid''); in such case this analysis becomes more complicated.

\newpage
\item[Cayley-Hamilton theorem:] a square matrix is always ``root'' of its characteristic polynomial (generalized to allow plugging in matrices instead of scalars $\lambda\in\RR$):\[
\chi_A(A) = 0 \in \RR^{n\times n}
\]

\item[Normal matrix:] square matrix that commutes with its transposed (/Hermitian for complex):
\[AA^* =A^*A\]
The following matrix types are normal: orthogonal (unitary), Hermitian (incl.\@ real symmetric; this also includes all (semi)definite types).

Property: the only normal matrices with eigenvalues on the unit circle are unitary.

\item[Orthogonal vectors:] two nonzero vectors $\bm x$, $\bm y$ are orthogonal when $\bm x\transp \bm y = 0$.

This is denoted as: $\bm x \perp \bm y$.

Two orthogonal vectors are linearly independent. A set of vectors is orthogonal if they are pairwise orthogonal (therefore, they form the basis of a subspace).

If also normalized (unit $2$-norm), they are called orthonormal.

\item[Orthogonal matrix:] real square matrix $Q$ such that $Q\transp = Q^{-1}$. A matrix $Q$ is orthogonal iff all columns (and rows) form an orthonormal basis.

\item[Unitary matrix:] Square matrix with $Q^*=Q^{-1}$ (Hermitian transpose). Generalization of orthogonal: if $Q$ is unitary and real, it is orthogonal.

Multiplication by orthogonal/unitary preserves inner product  (and therefore angles and norm-2 lengths):
\begin{gather*}
    (Q\bm x)^*(Q\bm y) = \bm x^*\bm y
\\    
\normtwo{Q\bm x} =
    \sqrt{(Q\bm x)^*(Q\bm x)} =
    \sqrt{\bm x^*\bm x} =
    \normtwo{\bm x}
\\    
\normfrob{Q\bm x} =
    \normfrob{\bm x}
\\[4pt]
|{\det}(Q)| =1 \qquad \text{(${\det}(Q)=\pm 1$, if $Q$ is real (orthogonal))}
\end{gather*}
Orthogonal/Unitary matrices are normal (commute with (Hermitian) transpose).

All eigenvalues $\lambda_i$ lie on the unit circle (these are the only normal matrices with this property). Note: even the real 2D rotations have complex eigenvalues!

Proof: let $Q$ be unitary and $\bm v$ be an eigenvector of $Q$ with eigenvalue $\lambda$, then
\[
Q\bm v = \lambda \bm v\implies (Q\bm v)^*  Q\bm v = (\lambda \bm v)^* \lambda \bm v
\implies \cdots \implies \norm{\bm v}^2 = |\lambda|^2 \norm{\bm v}^2
\]
\[
\implies \boxed{|\lambda| = 1}
\]

All orthogonal (unitary) transformations are isometries in $\RR^n$ ($\CC^n$).

Particular cases of orthogonal matrices (in $\RR$):
\begin{itemize}
    \item (Pure??) Reflection matrix: has one eigenvalue $\lambda_1=-1$, rest $\lambda_i=1$.
    \item Proper rotation matrix: for 2D, two complex conjugates in unit circle; for 3D, also the eigenvalue $\lambda_3=1$ (eigenvector is the axis of rotation); for higher dims, complicated (Special Orthogonal group, $\mathrm{SO}(n)$), but always $\det(Q)=+1$.
    \item Improper rotation matrix: composition of rotations and reflections that do not result in a proper rotation; $\det(Q)=-1$.
    \item Planar rotation: it is a proper rotation for which the matrix in some orthonormal basis is as follows:
    \begingroup
        \newcommand{\snth}{s_\theta}
        \newcommand{\csth}{c_\theta}
        \NiceMatrixOptions{code-for-first-row = \scriptstyle,code-for-first-col = \scriptstyle }
        \setcounter{MaxMatrixCols}{12}
        \newcommand{\blue}{\color{blue}}
        \[
        R_\theta(f) = \begin{pNiceMatrix}[last-row,last-col,nullify-dots,xdots/line-style={dashed,blue}]
        1& & & \Vdots & & & & \Vdots \\
        & \Ddots[line-style=standard] \\
        & & 1 \\
        \Cdots[color=blue,line-style=dashed]& & & \blue \csth &
        \Cdots & & & \blue {\text{-}\snth} & & & \Cdots & \blue \leftarrow p \\
        & & & & 1 \\
        & & &\Vdots & & \Ddots[line-style=standard] & & \Vdots \\
        & & & & & & 1 \\
        \Cdots & & & \blue \snth & \Cdots & & \Cdots & \blue \csth & & & \Cdots & \blue \leftarrow q \\
        & & & & & & & & 1 \\
        & & & & & & & & & \Ddots[line-style=standard] \\
        & & & \Vdots & & & & \Vdots & & & 1 \\
        & & & \blue \overset{\uparrow}{p} & & & & \blue \overset{\uparrow}{q} \\
        \end{pNiceMatrix}\]
        \[\snth = \sin\theta,\,\csth = \cos\theta\]
    \endgroup
    where $\theta$ is the rotation angle (counterclockwise), and $p$,~$q$ are the coordinate indices corresponding to a 2D plane.

    Every proper rotation can be written as composition (matrix product) of proper rotations [TODO: find source of this claim...].

\end{itemize}

\item[Semi-orthogonal/semi-unitary matrix:] matrix $\hat Q \in \RR^{m\times n}$ whose columns are orthonormal. Note that, by definition, $m \ge n$ (there cannot be a set of more than $n=m$ orthogonal vectors in $\RR^m$).

It has the property: $\hat Q\transp \hat Q = I$.

Note that, unless $m=n$, we don't have the property $\hat Q \hat Q\transp  \overset{!}{=} I$ (that would be fully orthogonal). (We see later that $\hat Q \hat Q\transp$ is a projection matrix).

Left and right multiplication by semi-orthonormal $\hat Q$ preserves 2-norm:
\[
    \normtwo{\hat Q \bm x}=
    \sqrt{(\hat Q \bm x)\transp (\hat Q \bm x)}=
    \sqrt{\bm x\transp \hat Q\transp \hat Q \bm x}=
    \normtwo{\bm x}
\]
\[
    \normtwo{ \bm x \hat Q}=
    \cdots =
    \sqrt{\smash[b]{{\hat Q\transp \underbrace{\bm x\transp\bm x}_{\text{scalar}} \hat Q}}}
    =\normtwo{\bm x}
\]\vspace{0.01em} % Fix underbrace

But multiplication by $\hat Q\transp$ does NOT preserve the 2-norm.



\item[Orthogonal components of a vector:] 
With the orthonormal set of vectors (basis of a subspace of $\RR^m$), forming the columns of the semi-orthogonal matrix $\hat Q$ \[
\hat Q=
\begin{pNiceMatrix}[vlines]
    \bm q_1 & \bm q_2 & \cdots &\bm q_n
\end{pNiceMatrix}
\in \RR^{m\times n}, \, m\ge n
,\] and a given vector $\bm u \in \sspan(\{\hat{\bm q_i}\}_{1\le i\le n} ) = \range(\hat Q)$, we can obtain the $i$-th coordinate on this basis by just using the scalar product:
\[
    (\bm u)_{i,(\hat Q)} = \bm q_i\transp \bm u
\]
For a general vector $\bm v$,  we obtain the ``residual component'' vector $\bm r$ (or simply ``residual''), with respect to $\hat Q$:
\[
    \bm r = \bm v - 
    \underbrace{\left(
        (\bm q_1^* \bm v)\, \bm q_1 + \cdots+
        (\bm q_n^* \bm v)\, \bm q_n
    \right)}_\text{projection of $v$ on $\sspan(\hat Q)$}
\]

It can be proven that $\bm r$ is either zero (when $\bm v\in \sspan(\hat Q)$) or orthogonal to $\sspan(\hat Q)$ (when $\bm v\notin \sspan(S)$), in both cases we have: \[\bm q_i^* \bm r = 0, \, \forall i\]

Then, for the expression:
\[
\bm v = \bm r + \sum_{i=1}^{n}(\bm q_i \bm q_i^*)\,\bm v
\]

If $\bm v \notin \sspan(Q)$ this is a decomposition into $n+1$ orthogonal components.

Otherwise ($\bm v \in \sspan(Q)$), $\bm r$ is zero; e.g.\@ if $m=n$, then $Q$ is a basis of the whole space $\RR^n$, so $\bm r$ must be zero.


\item[Orthogonal complement of a set of vectors:] 

Vector subspace formed by all vectors orthogonal to all vectors in a set $S$.

Denoted by: $S^\perp$.

Property: if $S$ is a subspace, then \[(S^\perp)^\perp = S.\]
And also, when $S\subseteq \RR^n$ is subspace, then
\[ S \oplus S^\perp = \RR^n.\]
[TODO: explain the meaning of a direct sum.]


\item[Diagonalization:] 
(a.k.a.\@ eigendecomposition) decomposition of a given matrix $A$ by using a \emph{similar} diagonal matrix $\Lambda$, that is:
\[\boxed{A = P\,\Lambda\,P^{-1}}.\] $\Lambda$ actually contains the eigenvalues ($\lambda_i$ repeated $\mu_a(\lambda_i)$ times), and the columns of $P$ are lin.\@ indep.\@ eigenvectors (they form a basis).

When such a decomposition is possible, that is, if $n$ lin.\@ independent eigenvectors can be found (hence $P$ square can be built), then $A$ is diagonalizable, a.k.a.\@ non-defective.

$A$ is diagonalizable iff all $\mu_a(\lambda_i)=\mu_g(\lambda_i)$, or iff (equivalent condition) $\sum_i\mu_g(\lambda_i)=n$.

(Note: here we assumed the convention that a complex eigenvalue is considered valid for real matrices)

In the case of defective matrices, this can be generalized by replacing $\Lambda$ diagonal with an ``almost-diagonal'' version $J$ called Jordan canonical form; $J$ is block diagonal, where each block $J_i$ has eigenvalues on the diagonal and ones on the first upper diagonal (or lower, depending on convention). If $A$ is diagonalizable, then $\Lambda=J$ is valid in the diagonalization.

Property: a matrix is \emph{orthogonally diagonalizable} ($A=Q\Lambda Q\transp$, with $Q\transp=Q^{-1}$) iff it is normal. If it is also real symmetric/Hermitian then all eigenvalues are real (in fact: for a normal matrix, the eigenvalues are real iff it is Hermitian).

\end{description}

\newpage
\addsec{Norms}

\begin{description}
\item[Vector norm axiomatic definition:]\phantom{ }

An operator on vectors $\norm{\cdot} : \RR^m \to \RR$ is called a norm when the following conditions are met:

    \begin{itemize}
        \item $\norm{a\bm x} = |a| \norm{\bm x}$ (homogeneity)
        \item $\norm{\bm x + \bm y} \le \norm{\bm x} + \norm{\bm y}$ (triangle inequality)
        \item $\norm{\bm x} = 0 \iff \bm x = \bm 0$
        
        (when only this one fails, we have a seminorm)
    \end{itemize}
\item[Vector norm properties:]\phantom{ }
    \begin{itemize}
        \item $\norm{\bm x} \ge 0$ (positive definiteness, sometimes as an axiom definition, but not necessary; also valid for seminorms)
        \item $
        |\norm{\bm x} - \norm{\bm y} | \le \norm{\bm x - \bm y}
        \quad \text{(follows from triangle inequality)}
        $
    \end{itemize}

\item[Vector $p$-norms:]\phantom{ }
\begin{align*}
    \norm{\bm x}_p =& \left(\sum_i |x_p|^p\right)^{1/p}
    \quad \text{($p$-norm)}
    \\
    \norm{\bm x}_1 =& \sum_i |x_i|
    \\
    \normtwo{\bm x} =& \left(\sum_i |x_p|^2\right)^{1/2}
    = \sqrt{\bm x ^* \bm x}
    \\
    \norminf{\bm x} =& \max_{1\le i\le m}|x_i|
\end{align*}

$\norminf{\bm x}$ can be proven to be the asymptotic case of $p$-norm for $p\to\infty$.

Note: not all vector norms are $p$-norms.

\item[Weighted norm:]\phantom{ }
\[
\norm{\bm x}_W = \norm{W\bm x}, \text{ for $W$ invertible (usually diagonal)}
\]

\item[H\"older's inequality:]

Relates scalar product and vector $p$-norms:
\[
|\bm x\transp \bm y| \le \norm{\bm x}_p\norm{\bm y}_q, \quad \text{ with }\frac{1}{p}+\frac{1}{q} = 1, \quad p, q\in [1,\infty]
\]

\item[Cauchy-Schwarz inequality:]
H\"older's for $p=q=2$.
\[
|\bm x\transp \bm y| \le \normtwo{\bm x}\normtwo{\bm y}
\]

Also recall: $
\bm x\transp \bm y =
    \normtwo{\bm x}\normtwo{\bm y} \cos \alpha_{\bm x,\bm y}$

\newpage
\item[Vector norms are equivalent:] (this is valid for finite-dimensional vector spaces)
\[
    c_1 \norm{\bm x}_\alpha \le
        \norm{\bm x}_\beta  \le
    c_2 \norm{\bm x}_\alpha
\]
For some $c_1$, $c_2$ that do not depend on $\bm x$ (but maybe on $n$).

Some particular cases:
\begin{align*}
    \normtwo{\bm x} &\le \normone{\bm x} \le \sqrt{n}\normtwo{\bm x}
    \\
    \norminf{\bm x} &\le \normtwo{\bm x} \le \sqrt{n}\norminf{\bm x}
    \\
    \norminf{\bm x} &\le \normone{\bm x} \le n\norminf{\bm x}
\end{align*}

\item[Spectral radius:] defined for square matrices as the largest eigenvalue (in abs.\@ value).
\[\rho(A) = \max_i |\lambda_i(A)| \]

\item[Matrix norms:]
The vector norm axiomatic definition can be directly generalized into matrices (just by replacing vectors with matrices).

Matrix norms are also all equivalent.

\item[``Entry-wise'' matrix norms:]
Norms that result from the use of a vector norm on the ``vectorized'' or ``flattened'' matrix $\bm v_A = \mathrm{vec}(A)$ (1D vector with $m\cdot n$ entries, may be traversed column-wise or row-wise or in any arbitrary order; this order irrelevant for most vector norms ``of interest'').

The usual example of ``Entry-wise'' matrix norm is the Frobenius norm (using vector 2-norm).

\item[Induced matrix norms:]
Given two vector norms $\norm{\cdot}_\alpha$ and $\norm{\cdot}_\beta$ that can be applied to the output and input spaces (resp.) of a matrix $A$, we define the induced matrix norm of $A$ as:
\[
\norm{A}_{(\alpha,\beta)}
=
\sup_{\bm x\neq \bm 0}
    \frac{\norm{A\bm x}_\alpha}{\norm{\bm x}_\beta}
=
\max_{\norm{\bm x}_{\beta}=1}
    {\norm{A\bm x}_\alpha}
\]

Pedantic remark: $\sup$ is not really needed and $\max$ is enough, because the set of values of the quotient is compact, regardless of the fact that $\RR^n-\{\bm 0\}$ is not compact.

Shorthand notation: $\norm{A}_{(\alpha,\alpha)}
=\norm{A}_{\alpha}$

For $Q$ orthogonal: $\norm{Q}_{2} = 1$, (because $Q\bm x$ preserves the 2-norm of $\bm x$).

Special cases:
\begin{itemize}
    \item $\displaystyle \norm{A}_1 = \norm{A}_{(1,1)} = \max_j \sum_i |A_{ij}|$
    \,\,\,
    (largest column, sum over rows)

    \item $\displaystyle \norm{A}_\infty = \norm{A}_{(\infty,\infty)} = \max_i \sum_j |A_{ij}|$
    \,\,\,
    (largest row, sum over columns)
    
    \item $\displaystyle \norm{A}_2 = \norm{A}_{(2,2)} = \sqrt{\rho(A\transp A)}= \sqrt{\rho(A A\transp)} = \sigma_1$
    
    Largest singular value of $A$ (see SVD), a.k.a. spectral norm.
\end{itemize}
\vspace{0.5em}

Property for any induced matrix norm $\alpha$: \[\rho(A) \le \norm{A}_\alpha\]

\newpage
\item[Frobenius norm:]

It is the ``entry-wise'' 2-norm.
\[
\normfrob{A}
    = \sqrt{\sum_i\sum_j |A_{ij}|^2}
    = \sqrt{\trace(A\transp\, A)}
    = \sqrt{\trace(A\, A\transp)}
\]

(See section on properties of $A\transp A$ and $AA\transp$)

And also (because singular values are eigenvalues of $A\transp A$):
\[
\normfrob{A}
    = \sqrt{\sum_i\sigma_i^2}
\]
(See section on SVD)

\item[Property:] for $Q_1$ and $Q_2$ orthogonal (of suitable sizes), the 2-norm and Frobenius norm are preserved:
\[\normtwo{Q_1A} = \normtwo{A} = \normtwo{AQ_2} \]
\[\normfrob{Q_1A} = \normfrob{A} = \normfrob{AQ_2} \]

\item[Submultiplicative norm:] Matrix norm $\norm{\cdot}$ with
\[
\norm{AB}\le \norm{A}\,\norm{B}
\]

All induced matrix norms (including $p$-norms) and the Frobenius norm are sub-multiplicative.

Example of NON-sub-multiplicative matrix norm: $\norm{A} = \max_{i,j} |a_{ij}|$ (this is the entry-wise $\infty$-norm).

\item[Property:]
$\norm{\cdot}$ is an induced matrix norm, then $\rho(A)<1\implies \norm{A} < 1$

(in fact, this would also work for any submultiplicative matrix norm??)

\item[Property:]
\[
    \lim_{k\to\infty}A^k = 0 \iff \rho(A) < 1
\]
% Exercise!
%Proof: choose any induced matrix norm $\norm{\cdot}$.
%Suppose $\rho(A)<1$, then $\norm{A} < 1$, then as $\norm{A^k}\le \norm{A}^k$, we have $\norm{A^k} \to 0$ and therefore $A^k \to 0$

\item[Property:]
For any submultiplicative matrix norm:
\[
    \lim_{k\to\infty}\norm{A^k}^{1/k} = \rho(A)
\]

\item[Compact set:]
In the usual case of a set in $\RR^n$ (or $\CC^n$), compact means ``closed and bounded''. (In the more general case, it means some topological abstract nonsense involving \emph{covers} and \emph{finite subcovers}).

\item[``Compactness argument'':] (a.k.a.\@ \emph{extreme value theorem})

``A continuous function attains its maximum (and minimum) value in a compact set''.


\end{description}

\newpage
\addsec{Properties of \texorpdfstring{$A\transp A$ and $AA\transp$}{AAT and AAT}}
What follows is important for the least squares methods and the Moore-Penrose pseudoinverse calculation.

More info in textbook [\footnote{\emph{{\'A}lgebra lineal y geometr{\'\i}a vectorial}, Borobia, A.\@ and Estrada. B.\@ UNED (Sanz y Torres), ISBN: 978-84-15550-85-3.}] Proposition 8.41 (I'm sorry, it is in Spanish...).

The matrix $A\transp A$ is sometimes called Gram matrix or Gramian matrix\footnote{See \url{http://www.seas.ucla.edu/~vandenbe/133A/lectures/inverses.pdf}.}.


Let $A\in \RR^{m\times n}$, then it holds:
\begin{itemize}
    \item $A\transp A$ has size $n \times n$ and $AA\transp$ has size $m \times m$.

    \item {} $A\transp A$ is symmetric square ($n\times n$) and is ``smaller'' than $A$ when $m>n$
    
    (e.g.\@ $A\bm x=\bm b$ is an overdetermined system).
    
    $AA\transp$ is symmetric square ($m\times m$) and is ``smaller'' than $A$ when $n>m$.
    
    \item {} As they are symmetric, both matrices are also \emph{normal}.

    \item {}$(A\transp A)$ and $(A A\transp )$ are symmetric positive semidefinite matrices (SPSD), therefore:
    \begin{itemize}
        \item All of the eigenvalues are nonnegative (can be zero).
        \item They admit the Cholesky decomposition (in the SPD case, it is also unique).
        %\item They admit the ``square root decomposition'': $M = B B$, with $B$ Hermitian (or real symmetric) and positive definite. Such $B$ (with those properties) is unique.
    \end{itemize}

    \item Each entry $(A\transp A)_{ij}$ is the inner product of two columns of $A$ (for $AA\transp$, those are rows of $A$). 
    
    This clearly explains why for $Q$ with $m=n$ orthonormal columns we get $ Q\transp Q=QQ\transp=I$; that is, $Q$ is orthogonal.
    
    Note that $Q$ here is square; if it was $\hat Q$ rectangular (i.e.\@ \emph{semi}-orthogonal, with orthonormal columns; requires $m\ge n$), then the product $\hat Q\transp \hat Q = I$ is never commutative (the resulting matrix has a different size!); in fact, $\hat{Q}\hat{Q}\transp$ is NOT full rank (nonetheless, it is still SPSD).
    
    As we will see later, $\hat{Q}\hat{Q}\transp$ is actually a projection matrix onto $\range(\hat Q)$.
    \item The trace is the sum of all elements of $A$ squared (square of Frob norm):
    \[ \trace(A\transp A) = \trace(A A\transp) = \sum_i\sum_j |a_{ij}|^2= \normfrob{A}^2\]
    \vspace{-0.55cm}

    \item $\rank(A) = \rank(A\transp A) = \rank(AA\transp )$.
    \item $\Ker(A) = \Ker(A\transp A)$.
    \item $\Ker(A\transp) = \Ker(AA\transp )$.
    \item If $m < n$, $A\transp A$ cannot be invertible.
    \item In the case $m\ge n$, $(A\transp A)$ is invertible iff $A$ is full (column) rank.

    \item $(A\transp A)$ and $(A A \transp)$ share the same nonzero eigenvalues.

    \item Being symmetric, by the spectral theorem, $A\transp A$ admits orthogonal diagonalization (and with real eigenvalues): \[A\transp A = V\Lambda V\transp.\] And also $A A \transp = U\Lambda^{\prime} U\transp$, where $\Lambda$ and $\Lambda^{\prime}$ contain the same nonzero elements.
    
    \item If $A$ is square, the largest singular value is $\normtwo{A} = \sqrt{\rho(A\transp A)}$ (see SVD section).

\end{itemize}

\addsec{Singular Value Decomposition (SVD)}

Best explanation ever: \url{https://www.youtube.com/watch?v=rYz83XPxiZo}

Also, worth reading:
\begin{itemize}
    \item \url{https://gregorygundersen.com/blog/2018/12/10/svd/}
    \item \url{https://gregorygundersen.com/blog/2018/12/20/svd-proof/}
\end{itemize}
\vspace{0.5em}

Consider a general matrix $A \in \RR^{m\times n}$ (could be not full rank and rectangular (not square)).

[TODO: add geometric definition (ellipsoid axis and radius)]

We are looking for two sets of orthonormal vectors $\bm u_i \in \RR^{m}$ and $\bm v_i \in \RR^n$, such that:
\[A\bm v_i = \sigma_i \bm u_i\,, \quad \sigma_i \ge 0\]
And also $\sigma_i$ are ordered in descending order:
\[ \sigma_i > \sigma_j \iff i > j \]

\begin{itemize}
    \item $\bm u_i$ are called left singular vectors, they are in the image (output space).
    \item $\bm v_i$ are called right singular vectors, they are in the domain  (input space).
    \item $\sigma_i \ge 0$ are called singular values.
\end{itemize}

This could be seen as the generalization of the definition of eigenvalue/vector for non-square matrices; that is, for an eigenvector we would have $\bm u_i= \bm v_i$, but note that this is not a proper generalization because an eigenvalue can be negative or even complex (SVD has all $\sigma_i$ real and nonnegative) and the eigenvectors may not form an orthogonal basis (SVD has all $\bm u_i$ orthonormal and all $\bm v_i$ orthonormal). Moreover (proven later) a SVD always exists regardless of whether $A$ square is diagonalizable or not.

We can find at most $N=\min(m,n)$ independent vectors and the first $r=\rank(A)$ of which have $\sigma_i\neq 0$ and the last $N-r$ have $\sigma_i= 0$:
\[
\left\{
\begin{array}{lclcl}
 A \bm v_1 &=& \sigma_1 \, \bm u_1 \\
 A \bm v_2 &=& \sigma_1 \, \bm u_2 \\
 &\vdots \\
 A \bm v_r &=& \sigma_r \, \bm u_r \\
 A \bm v_{r+1} &=& \bm 0 &=& 0 \,\bm u_{r+1} \\
 &\vdots \\
 A \bm v_N &=& \bm 0 &=& 0 \,\bm u_{N} \\
\end{array}
\right.
\]
\newpage

Now, from these equations:
\begin{itemize}
    \item The last $\bm v_i$ (for $i>r$) must form an orthonormal basis of $\Ker(A)$.

    \item The first $\bm u_i$ (for $i\le r$) must form an orthonormal basis of $\range(A)$.

    \item The last $\bm u_i$ (for $i>r$) can be chosen arbitrarily, but in such a way that all $\bm u_i$ end up being orthonormal. Besides [TODO: check this], in the case that $n\ge m$, these last $\bm u_i$ span the subspace $(\range(A))^\perp$ (orthogonal complement of the range).

\end{itemize}


This can be rewritten in matrix notation instead of vector-by-vector:
\[
A \,
\begin{pNiceMatrix}[vlines]
    \bm v_1 & \cdots&\bm v_r& \cdots & \bm v_N \\
\end{pNiceMatrix}
=
\begin{pNiceMatrix}[vlines]
    \bm u_1 & \cdots&\bm u_r & \cdots & \bm u_N \\
\end{pNiceMatrix}
\begin{pNiceMatrix}
    \sigma_1 \\
    & \sigma_2 \\
    & & \ddots \\
    & & & \sigma_r \\
    & & & & \ddots \\
    & & & & & \sigma_N
\end{pNiceMatrix}
\]
\[
\implies
A \hat V = \hat U \hat\Sigma
\implies
\boxed{A = \hat U \hat \Sigma \hat V\transp}
\]

With $\hat{U}\in \RR^{m\times N}$, $\hat{V}\in \RR^{n\times N}$, $\hat\Sigma\in\RR^{N\times N}$.

$\hat U$ and $\hat V$ are \emph{semi}-orthogonal.
$\hat\Sigma$ is square diagonal with nonnegative entries in descending order (the $N-r$ last entries are zero, $\sigma_i=0$).

The columns of $\hat U$ span $\range(A)$.

Note that all of this is true regardless of the shape of $A$ (given by $m,n$; and recall that $N=\min(m, n)$).

When $\hat U$ or $\hat V$ is square, we write $U=\hat U$ or $V=\hat V$, and then $U$ or $V$ is orthogonal (and not only \emph{semi}-orthogonal).

Depending on $m,n$ we have that:
\begin{itemize}
    \item $m=n=N$: $A$ is square; then both $U=\hat{U}$ and $V=\hat{V}$ are also square.
    
    Also $\hat\Sigma$ has the same size as $A$, then we write $\hat\Sigma=\Sigma$.
    
    Hence, we can write: \[\boxed{A=U\Sigma V\transp}\]
    \item $m>n=N$: $A$ is \emph{tall}; then $V=\hat V$ is square ($m\times m$) and $\hat U$ has the same size as $A$ (also \emph{tall}).
    
    
    Hence, we can write: \[\boxed{A=\hat U\hat\Sigma V\transp}\]
    \item $n>m=N$: $A$ is \emph{fat}; then $U=\hat U$ is square ($n\times n$) and $\hat V\transp$ has the same size as $A$ (note that it is $\hat V\transp$ and not $\hat V$ (!); the rows of $V\transp$ are right singular vectors).
    
    This case seems problematic, because we have that $\hat{V}\hat{V}\transp\neq I$ (but $\hat{V}\transp\hat{V}=I$).
    
    But, we can write instead: \[
         \hat{V}\transp A\transp=\hat\Sigma\transp U\transp
         \implies
         A\transp=\hat V\hat\Sigma\transp U\transp
         \implies
         \boxed{A= U\hat\Sigma \hat V\transp}
    \]
\end{itemize}

Now, assume that $A=\hat U\hat\Sigma \hat V\transp$ is a valid SVD of $A$, then $A\transp = \hat V\hat\Sigma \hat U\transp$ is a valid SVD of $A\transp$ (just by swapping $\hat U$ and $\hat V$).

This is the reason why we usually focus on the case $m\ge n$: the case $m < n$ can be analyzed with the transposed matrix and then swapping left and right singular vectors (singular values are the same).

\subsection*{SVD as a sum of outer products}
Because $\hat\Sigma$ is diagonal, we can write:
\begin{align*}
A &= \hat U \hat \Sigma \hat V\transp
=
\begin{pNiceMatrix}[vlines]
    \bm u_1 & \cdots&\bm u_r & \cdots & \bm u_N \\
\end{pNiceMatrix}
\begin{pNiceMatrix}
    \sigma_1 \\
    & \sigma_2 \\
    & & \ddots \\
    & & & \sigma_r \\
    & & & & \ddots \\
    & & & & & \sigma_N
\end{pNiceMatrix}
\begin{pNiceMatrix}[hlines]
    \bm v_1\transp \\ \vdots\\\bm v_r\transp\\ \vdots \\ \bm v_N\transp \\
\end{pNiceMatrix}
=
\\
A&=
\begin{pNiceMatrix}[vlines]
    \bm u_1 & \cdots&\bm u_r & \cdots & \bm u_N \\
\end{pNiceMatrix}
\begin{pNiceMatrix}[hlines]
    \sigma_1\bm v_1\transp \\ \vdots\\\sigma_r\bm v_r\transp\\ \vdots \\ \sigma_N\bm v_N\transp \\
\end{pNiceMatrix}
=
\sum_{i=1}^N \bm u_i\, \sigma_i \bm v_i\transp
=
\sum_{i=1}^N \sigma_i (\bm u_i\,  \bm v_i\transp)
\end{align*}

Also, because $\sigma_i=0$ for $i>r=\rank(A)$:
\[
    \boxed{
        A = \sum_{i=1}^r \sigma_i (\bm u_i\,  \bm v_i\transp)
    }
\]

\subsection*{Full SVD}

The described SVD is actually already a reduced version (called ``thin SVD'' in Wikipedia), because $\hat\Sigma$ is square and, if $A$ is not square, then $\hat\Sigma$ it is ``smaller'' than $A$.

If $A$ is rectangular we can force $\hat\Sigma$ to have the same size as $A$ and have both $U$ and $V$ square (sizes $m\times m$ and $n\times n$ respectively); now both result orthogonal instead of just one of them orthogonal and the other \emph{semi}-orthogonal.

This is done as follows:
\begin{itemize}
    \item If $m>n$: add $m-n$ zero rows to $\hat\Sigma$ and $m-n$ arbitrary orthonormal columns to $\hat U$; now, the resulting matrix $U$ is square and orthogonal (spans $\RR^m$ instead of just $\range(A)$).
    \item If $m<n$: add $n-m$ zero columns to $\hat\Sigma$ and $n-m$ arbitrary orthonormal columns to $\hat V$ (rows to $\hat V\transp$); now, $\hat V=V$ becomes square and orthogonal.
\end{itemize}

With that, we obtain the ``Full SVD'' decomposition:
\[ \boxed{A = U \Sigma V\transp} \]

In the full SVD there are always $m$ left singular vectors ($\bm u_i$, $U\in\RR^{m\times m}$) and $n$ right singular vectors ($\bm v_i$, $V\in\RR^{n\times n}$); whereas in the reduced (thin) SVD there are $N=\min(m,n)$ left and right singular vectors. In both cases, there are $N$ singular values.

In SciPy (\verb|scipy.linalg.svd|) we can choose:
\begin{itemize}
    \item \verb|full_matrices=True| (default): Full SVD described here ($\Sigma$ and $A$ have the same shape; and $U$, $V$ are always square).
    
    Same as the default in Matlab.
    \item \verb|full_matrices=False|: the SVD described above (with square $\Sigma$; and $U$, $V$ are not always square).
    
    Same as the \verb|'econ'| variant in Matlab.
\end{itemize}
Note that the ``compact SVD'' described next is not provided by SciPy, because that would require infinite precision on the singular values that are exactly zero.

\subsection*{Compact SVD}

For simplicity, assume the ``tall matrix'' case $m\le n$ ($N=\min(m,n)=n$).

In the last SVD equations (for $i>r$), we needed to compute vectors $\bm v_i$ in $\Ker(A)$. But these are unnecessary in order to recover $A$ from the SVD.

Indeed, knowing that $\sigma_i=0$ for $i>r$, the reduced SVD (thin) can be written as:
\[
    A{V} = \hat{U} \hat{\Sigma} \iff A = \hat{U} \hat{\Sigma}V\transp
    =\sum_{i=1}^{n}
        \bm u_i \, \sigma_i \, \bm v_i\transp
    =\sum_{i=1}^{r}
        \bm u_i \, \sigma_i \, \bm v_i\transp,
\quad r=\rank(A)
\]

Then, define $\hathat U$ and $\hathat V$ as the same matrices as $\hat U$ and $V=\hat V$, but removing the last $n-r$ columns of both (rows of $V\transp$). Also define $\hathat \Sigma\in\RR^{r\times r}$ as the square diagonal matrix with the first $r$ singular values $\sigma_i$ (the nonzero ones).

Then, from the last expression, we can simply write:
\[
    A = \hathat U \hathat\Sigma \hathat V\transp
\]

This is called the ``compact'' SVD. In practice, no numerical library implements this, because it would require to compute the zero singular values with exact precision.

\subsection*{Lossy compression using truncated SVD}

We can remove \emph{even more} terms in the sum of outer products, but then an equality cannot be achieved, instead we get an approximate decomposition for which we can obtain error bounds. This is called ``truncated'' SVD and can be used to build a lossy compression algorithm for matrices (e.g.\@  rasterized images); note that this is rarely used in practice for image compression.

Assume that we keep the $R\le r=\rank(A)$ first terms, then we have:
\[
     A \approx A_R = \sum_{i=0}^{R} \bm u_i \, \sigma_i \, \bm v_i\transp
\]

Now, note that the matrix difference (error) is:
\[
A - A_R = \sum_{i=R+1}^{r} \bm u_i \, \sigma_i \, \bm v_i\transp
\]

Then, it can be proved that:
\[
\boxed{\normtwo{A - A_R} = \sigma_{R+1}}
\]

And also:
\[
\normfrob{A - A_R} = \sqrt{\sum_{i=R+1}^r\sigma_{i+1}^2}
\]

Now, compression is only controlled by the choice of $R$ (we ignore other possible numerical errors in the SVD algorithm, like quantization): this $R$ can fixed or chosen by dropping the terms with singular values below a relative tolerance. This tolerance can be defined with respect to the largest singular value $\sigma_1$ (2-norm) or using the Frobenius norm as reference (constrain on ``energy loss'').


Recall that $\bm u_i$ and $\bm v_i$ consist of $m$ and $n$ numbers (resp.), so, also counting $\sigma_i$ we need to store $(m+n+1)$ numbers for each tuple $(\bm u_i,\,\bm v_i,\,\sigma_i)$ (used to build a term in the sum). Therefore, we require to store $R\cdot (m+n+1)$ numbers in total.

As there are $m\cdot n$ elements in $A$ we may define the data compression ratio or efficiency as:
\[
\eta
= \frac{\text{uncompressed size}}{\text{compressed size}}
= \frac{m\cdot n}{R\cdot (m+n+1)}
\]

Note: it can happen that $\eta\le 1$, then this ``compression'' is useless (more efficient to just store $A$).

\subsection*{Existence and uniqueness of the SVD}

\subsubsection*{Proof: Existence of SVD (from G.\@ Strang lecture)}
[TODO: finish this]

% Assume $A\in\RR^{m\times n}$ full rank.

% Let us define the $\bm u_i$ in terms of $\sigma_i$ and $\bm v_i$ so that if both are found, $\bm u_i$ is unambigously defined: $\bm u_i = \frac{A \bm v_i}{\sigma_i}$; once $\sigma_i$ are defined sorted in $i$; with all this, the only ``degree of freedom'' is on the choice of $\bm v_i$.

% Choose $\bm v_i$ and $\sigma_i^2$ as the eigenvectors/values in the orthogonal diagonalization: \[A\transp A= V\Lambda V\transp\]

% Then: 

% Let's prove that those resulting $\bm u_i$ are orthogonal, pick WLOG $\bm u_1$, $\bm u_2$:
% \[ \bm u_1\transp \,\bm u_2 =
% \frac{(A \bm v_1)\transp}{\sigma_1}
% \,
% \frac{(A \bm v_2)\transp}{\sigma_2}
% =
% \frac{
%     \bm v_1\transp
%     A\transp A
%     \bm v_2
% }{\sigma_1\sigma_2}
% \]
% We know that $\bm v_2$ is eigenvector of $A\transp A$: 
% \[ \bm u_1\transp \,\bm u_2 =
% \frac{
%     \bm v_1\transp
%     \sigma_2^2
%     \bm v_2
% }{\sigma_1\sigma_2}
% =
% \frac{\sigma_2}{\sigma_1}
% \bm v_1\transp
% \bm v_2 =
% \bm 0
% \]

% Steps: $V\transp$ rotation, $\Sigma$ axis stretching, $U$ rotation. We can have $U=V$ in the case that $A$ is symmetric AND positive definite.    

% From SVD we can easily get a SQ decomposition (Symmetric-by-Orthogonal).

% [TODO: ...]

\subsubsection*{Proof: Existence of SVD (from LTH lectures, same as Trefethen \& Bau book)}

[TODO: finish this]

% We have $\sigma_1 = \normtwo{A}$ (see properties), then $A \bm v_1 = \sigma \bm v_1$.

% Now extend $\bm v_1$ to form $V_1$:
% \[
% A V_1 =
%     [A \bm v_1,  A \bm v_2, ..., A \bm v_{?}]
%     =
%     [\sigma_1 \bm u_1,  A \bm v_2, ..., A \bm v_{?}]
% \]
% Now premultiply by $U^*$, In particular, the first resulting row:
% \[
% U_1^* (\sigma_1 \bm u_1)
% =
% \begin{pNiceMatrix}[hlines]
%     \bm u_1^* \\ \bm u_2^* \\ ... \\ \bm u_{?}^*
% \end{pNiceMatrix}^* \sigma_1 \bm u_1
% =
% \sigma_1 \begin{pNiceMatrix}[hlines]
%     \bm u_1^*\bm u_1 \\ \bm u_2^*\bm u_1 \\ ... \\ \bm u_{?}^*\bm u_1
% \end{pNiceMatrix}
% =
% \sigma_1 \begin{pNiceMatrix}[hlines]
%     1 \\ 0 \\ ... \\ 0
% \end{pNiceMatrix}
% =
% \sigma_1 \bm e_1\]

\subsubsection*{Proof: Uniqueness of SVD}
[TODO: finish this]

% Let us consider the orthogonal diagonalization of SPSD matrix $A\transp A$:
% \[
%     A\transp A = V_* \Lambda V_*\transp \in \RR^{n\times n}
% \]

% and also $AA\transp$ (also SPSD):
% \[
%     AA\transp = U_* \Lambda^\prime U_*\transp \in \RR^{m\times m}
% \]

% Now, because of the properties of these matrices, $\Lambda$ and $\Lambda^\prime$ are diagonal and contain the same nonzero elements: eigenvalues of  $A\transp A$ and $AA\transp$, which must be nonnegative (by SPSD).

% Consider the full SVD of $A$: $A=U\Sigma V\transp$.

% Then: \[A\transp A
%     = (U\Sigma V\transp)\transp U\Sigma V\transp 
%     = V\transp\Sigma\transp \underbrace{(U\transp U)}_I\Sigma V\transp 
%     = V\Sigma_1^2 V\transp 
% \]
% Note that, in general, $\Sigma$ is not square (has same shape as $A$), so we define here $\Sigma_1^2=\Sigma\transp \Sigma$ as the diagonal square matrix of size $n\times n$ with the diagonal of $\hat\Sigma$ squared. Note also that, in the case $n<m$, there are extra zeros at the end of the diagonal.

% Now, repeat the same with $AA\transp$:
% \[AA\transp 
%     = \cdots
%     = U\Sigma_2^2 U\transp 
% \]

% $\Sigma_1^2\in\RR^{n\times n}$ and $\Sigma_2^2\in\RR^{m\times m}$ are both square and diagonal, one of them may have extra zeros at the end of the diagonal when $m\neq n$, and the rest of the diagonal is the same.

% Comparing these expressions with the orthogonal diagonalizations, we see that those are orthogonal diagonalizations, for $V=V_*$ and $U=U_*$.

% Therefore: $A=U\Sigma V\transp$ is SVD iff we have the above eigendecompositions with $V=V_*$ and $U=U_*$ and $\Lambda$, $\Lambda^\prime$ containing all the squares of the singular values (one of them, possibly with extra zeros).

% Therefore (ignoring the possible extra zeros) $\sigma_i^2$ are the eigenvalues of both $AA\transp$ and $A\transp A$.

% The question of the uniqueness of SVD is then equivalent to the uniqueness of these orthogonal diagonalizations.

[TODO: finish this; the argument is that there is only ambiguity when there are multiple eigenvalues; otherwise the eigenspaces are all dimension 1 and the sign of the eigenvectors is the only degree of freedom (in the complex case, the choice is in the whole unit circle instead of just a sign ).]


% In the SVD we are looking for $\bm v_i$, $\bm u_i$ and $\sigma_i$, such that:
% \[
%     A\bm v_i = \sigma_i \bm u_i, \quad i\in\{1, ..., N\}, \quad N = \min(m,n)
% \]

% \subsection*{Full SVD}
% % Now we assume the case $m\ge n$ (thin matrix $A$).

% If we complete the basis in the domain (used in the full SVD) we need to use $\sigma_i=0$, so we forming a base of the null space:
% \[
%     A \bm v_i = \bm 0, \text{ for } i > r
% \]


% Then:
% \[
% A\transp A =
%     V \Sigma\transp U\,
%     U\transp \Sigma  V\transp
%     =
%     V (\Sigma\transp \Sigma)  V\transp
% \]

% This can be seen as an eigendecomposition of $A\transp A$: $\Sigma\transp \Sigma = (\sigma_i^2)$ the eigenvalues, and $\bm v_i$ are the corresponding eigenvectors.

% Similarly:
% \[
% A A\transp =
%     \cdots
%     =
%     U \Sigma\Sigma\transp U\transp
% \]
% Then, $\bm u_i$ are the eigenvectors of $A A\transp$ and the eigenvalues are again $\sigma_i^2$. (We will ignore this part).

% If $\sigma_i$ is repeated we have ambiguity in the choice of the basis.



% `[Left and right singular vectors, singular values]

% [Question: case n > m ->
%  if in m > n need to add columns to U and rows to Sigma,
%  should we add instead rows to VH and cols to Sigma??]

% [Reduced SVD/thin SVD/etc):
% Study shape of U and Sigma (square when?)
% ]

% [
% Lec3
% --> SVD as change of bases
% ]

% [If A is full rank:
%  -> Overdetermined system is incompatible.
%  -> Underdetermined system is comp. indeterminate (infinite solutions)
% ]

% [
% Number of nonzero singular values = rank(A)
% ]

% [
% $Range(A) = span{u_1, ..., u_r}$
% $Null(A) = span{v_{r+1}, ... v_{n}}$
% ]

% [
% Prop:

% For SVD $U\Sigma V\transp$, the elements in $\Sigma$ are indeed the singular values of $A$ [proof (by induction??)].
% ]


% [
% Fundamental property:
% [if m>=n] Singular values of A are the Eigenvalues of ATA.

% [if n<m] use AAT
% ]


\newpage
\addsec{SVD properties}

\begin{itemize}
\item
If $D$ is diagonal with nonnegative values in descending order, its SVD is simply $I D I$.
\item
If $D$ is diagonal with nonnegative values, its SVD is $P D_{*} P\transp$, with $D_*$ same as $D$ but with the diagonal in descending order, and $P$ a permutation matrix (is orthogonal).
\item
If $D$ is diagonal and real, its SVD is $P_1 D_{*} P_2\transp$, with $D_*$ same as $D$ but with the absolute values of the diagonal of $D$ in descending order, and $P_1$, $P_2$ are generalized permutation matrices that can have $-1$ instead of $+1$ (can be proven orthogonal).
\item
(Consequence of orthogonal/unitary invariance of  $U$ and $V\transp$):
\[\normtwo{A} = \normtwo{\Sigma} = \sigma_1 = \max \sigma_i\]
\[\normfrob{A} = \normfrob{\Sigma} = \sqrt{\sum_i \sigma_i^2}\]
\item
$
    \rank(A) =
    \rank(\Sigma) =
    \text{number of nonzero $\sigma_i$}
$
\item $A$ is full rank iff all singular values are nonzero ($\rank(A) = \min(m,n)$).

\item
The nonzero singular values $\sigma_i$ of $A$ are the square roots of the eigenvalues of $A^*A$ and $AA^*$. Proof: just expand $A^*A$ with the SVD of $A$ and obtain a similarity relation.

\item
The singular values $\sigma_i$ of $A$ are exactly the eigenvalues of $A^*A$ (if $m\ge n$) or $AA^*$ (if $n\ge m$).


\item If $A$ is square: $|{\det}(A)| = |{\det}(\Sigma)| = |\prod \sigma_i| = \prod \sigma_i$

\item
If $A=A^*$ (Hermitian, including real symmetric), then $\sigma_i = |\lambda_i|$.

Proof: symmetric is always diagonalizable (orthogonally and with real eigenvalues), therefore: \[
    A
    = Q\Lambda Q^* =Q \,\mathrm{abs}(\Lambda) \mathrm{sgn}(\Lambda)\, Q^*
    = Q\, \mathrm{abs}(\Lambda) \,(Q\,\mathrm{sgn}(\Lambda))^*
\]
We can get a valid a SVD out of this, because:
\begin{itemize}
    \item {} $Q\,\mathrm{sgn}(\Lambda)$ is unitary
    \item {} The descending order of $\sigma_i$ because the order of $\lambda_i$ can be chosen arbitrarily in the diagonalization (because permutation matrices are unitary).
\end{itemize}

\item If $A$ is symmetric positive (semi)definite (SPSD), then the SVD with $U=V$ exists and is a valid diagonalization. Reason: SPSD is normal with nonnegative eigenvalues, and the eigenvalues in the diagonalization can be chosen in descending order.

\item Let $r=\rank(A)$, then 
\[
   \left\{
   \begin{array}{lcl}
        \sspan(\bm v_{r+1}, \,...,\, \bm v_{n}) &=& \Ker(A)
        \\
        \sspan(\bm u_1, \,...,\, \bm u_r) &=& \range(A)
        \\
        \sspan(\bm u_{r+1}, \,...,\, \bm u_{m}) &=& \range(A)^\perp
   \end{array}
   \right.
\]


\item Because $\Sigma$ is diagonal, the SVD can be written as a sum of outer products:
\[
A = U\Sigma V\transp = \sum_{i=1}^N \bm u_i \sigma_i \bm v_i\transp
\]

% \item 

\end{itemize}

\subsection*{Solving linear systems using SVD}

Assume that the system $A\bm x=\bm b$ has at least one solution (for the no-solutions case, use least squares method).

\subsubsection*{Overdetermined linear system}

Assume $m>n$. Consider the change of bases given by the full SVD.
\[
A\bm x = \bm b; \quad A = U\Sigma V\transp; \quad
\overline{\bm b} = U\transp \bm b;\quad \overline{\bm x} = V\transp \bm x
\]
\[
A\bm x = \bm b \implies
\Sigma V\transp {\bm x} = U\transp\, {\bm b}
\implies
\boxed{\Sigma \overline{\bm x} = \overline{\bm b}}
\]

Because $m>n$:
\[
\Sigma = \begin{pNiceMatrix}
    \Sigma_1\\0_{(m-n)\times n}
\end{pNiceMatrix}
\]

For the case $A$ full rank (implies $\Sigma_1^{-1}$ exists), there exists a solution iff:
\[
\overline{\bm b} = \begin{pNiceMatrix}
    \overline{\bm b}_1\\ \bm 0
\end{pNiceMatrix}
\]

\subsubsection*{Underdetermined linear system}

Assume $m<n$.
\[
A\bm x = b
 \implies
  \Sigma \overline{\bm x}
  = \overline{\bm b}
\]
\[
\Sigma=\begin{pNiceMatrix}
    \Sigma_1 & 0_{m\times (n-m)}
\end{pNiceMatrix}
\]

For the case $A$ full rank (implies $\Sigma^{-1}$ exists) the nonunique solution is given by:
\[
\overline{\bm x} = \begin{pNiceMatrix}
    {\Sigma}_1^{-1}\, \overline{\bm b}\\ \bm t
\end{pNiceMatrix}
\]

For all $\bm t \in \RR^{n-m}$.

\newpage
\addsec{Projections}

A projector or projection is an idempotent matrix/lin.\@ operator
(must be square):
\[
    P = P^2
\]
Elements in $\range(P)$ are kept invariant (are eigenvectors with $\lambda_i=1$):
\[
    \bm x\in\range(P) \implies
    P \bm x^\prime = \bm x \implies
    P^2 \bm x^\prime = P \bm x \implies
    P \bm x^\prime = P \bm x \implies
    \bm x = P \bm x
\]
Note that this does \emph{not} imply that $\bm x = \bm x^\prime$ or that $\bm x^\prime\in \range(P)$; but only that, once applied, the subsequent applications of $P$ leave $\bm x$ intact.

Visual idea: ``cast'' a light on a point vector $\bm v$ to produce a ``shadow'' ``onto'' the surface of the range subspace. The ``shadow'' is $P\bm v$, called projection of the vector $\bm v$.

We say that $P$ projects ``onto'' $\range(P)$ ``along'' $\Ker(P)$.

The projection of the ``projection error''/``residual component $\bm r$'' is zero (you cannot project anything that could not be projected before). Proof:
\[
    P(P\bm v -\bm v) = P^2 \bm v - P\bm v = P\bm v - P\bm v = \bm 0
\]

Thus: \[\boxed{P\bm v - \bm v \in \Ker(P)}\]

We can interpret $P\bm v - \bm v$ as the segment that joins the point $\bm v$ and its projection $P \bm v$, and, depending on $P$, it may or may not be perpendicular to the ``surface'' $\range(P)$. % In fact, the direction of $P\bm v - \bm v$ may not be constant for different $\bm v$.

All the points on the line that joins $\bm v$ with $P\bm v$ end up in $P\bm v$. Those are the vectors of the form $\bm w = P \bm v + \lambda (\bm v - P\bm v),\, \forall \lambda\in\RR$. Proof: 
\[
P\bm w =
    P(P \bm v + \lambda (\bm v - P\bm v))
    = P \bm v + \lambda (P\bm v - P\bm v)
    = P \bm v
\]

Complementary projection: $I-P$.
\[
    \text{It is a valid projection: $(I-P)^2 = I-P-P+P^2 = I-P$.}
\]
$I-P$ projects along $\range(P)$ onto $\Ker(P)$:
\begin{gather*}
\Ker(P) = \range(I-P) \\
\Ker(I-P) = \range(P)
\end{gather*}

Proof:
\begin{itemize}
    \item (for $\bm v\neq 0$) $\bm v \in \Ker(P) \implies P\bm v = \bm 0 \implies
    (I-P)\bm v = \bm v - P\bm v= \bm v - \bm 0\neq \bm 0$
    \item $\bm v^\prime\in \range(I-P) \implies \bm v^\prime = (I-P)\bm v = \bm v - P\bm v \implies P \bm v^\prime = \bm 0 \implies \bm v^\prime \in \Ker(P)$
\end{itemize}

There is no common nonzero vector: $\Ker(P)\cap\range(P) = \{\bm 0\}$. Proof: \[(I-P)\bm v = P\bm v = \bm 0 \implies \bm v - \bm 0 = \bm 0.\]

Then $P$ and $I-P$ divide $\RR^n$ into two complementary subspaces $S_1 \cap S_2 = \{\bm 0\}$, as in the direct sum: $S_1 \oplus S_2$. Therefore, they decompose each vector uniquely in sum of two vectors that are in $S_1$ and $S_2$: \[\bm s = \bm s_1 + \bm s_2\]

Proof:
\begin{gather*}
   \bm v = \bm v_1 +\bm v_2  = \bm v_1^\prime + \bm v_2^\prime 
   \\
   \left\{
   \begin{aligned}
        P\bm v &= P(\bm v_1 +\bm v_2) =  \bm v_1 + \bm 0 = \bm v_1
        \\
        P\bm v &= P(\bm v_1^\prime + \bm v_2^\prime ) =  \bm v_1^\prime + \bm 0 = \bm v_1^\prime
   \end{aligned}
   \right\}
   \implies  \bm v_1  = \bm v_1^\prime 
   \implies \bm v_2  = \bm v_2^\prime 
\end{gather*}

And, conversely, any two complementary subspaces (their direct sum is $\RR^n$) define a pair of complementary projections.

If $S_1 = S_2^\perp$ (subspaces with pairwise orthogonal vectors), then $P$ is called ``orthogonal projection''. (Note that $P$ is almost always not an ``orthogonal matrix'' itself, because it is generally not invertible).

It can be proven that a projection $P$ is orthogonal iff:
\[\boxed{P = P\transp} \quad (P = P^*\text{ for complex spaces).}\]

The orthogonal projection ``onto'' a single vector $\bm u$ (1D subspace) is defined by its outer product:
\[
P_{\bm u} = 
\frac{\bm u \bm u\transp}{\bm u\transp\bm u}
= \frac{\bm u \bm u\transp}{\displaystyle\normtwo{\bm u}^2} =
    \hat{\bm u} \hat{\bm u}\transp
\]

The orthogonal projection ``along'' a single vector $\bm u$ is then:
\[
P_{\bm u^\perp} = I - \hat{\bm u} \hat{\bm u}\transp
\]
Alternative expression (useful for QR decomposition):
\[
\bm x \mapsto
P_{\bm u^\perp}\, (\bm x)
=
    \left(
        I - \frac{\bm u \bm u\transp}{\bm u\transp\bm u}
    \right)\,\bm x
=
        \bm x - \underbrace{\left(
        \frac{\bm u\transp \bm x}{\bm u\transp\bm u}
        \right)}_{\text{scalar}}\, \bm u
\]

Orthogonal projection ``onto'' vector subspace spanned by orthonormal basis (columns of matrix $\hat{Q}$):
\[ 
\hat Q = (\bm q_1, \bm q_2, \dots, \bm q_r) \implies
\boxed{
    \bm v \mapsto P \bm v= \sum_{i=1}^r (\bm q_i \bm q_i \transp) \bm v
}\implies 
\boxed{
    P = P_{\range(\hat Q)} = \hat Q \hat Q\transp
}
\]

Orthogonal projection ``onto'' range of an arbitrary full column rank matrix $A$:
\[
\boxed{
    P_{\range(A)} = A (A\transp A)^{-1} A\transp
}
\]
The matrix $A^\dagger = (A\transp A)^{-1} A\transp$ is the Moore–Penrose pseudoinverse (for the full column rank case), and is a \emph{left} inverse, meaning that $A^\dagger A = I$.

Therefore: \[ 
\boxed{
    P_{\range(A)}
    = A A^\dagger
}\]

\addsec{QR factorization}

Goal: decompose $A\in\RR^{m\times n }$ as
\[
    A = QR
\]
with $Q$ orthogonal (square) and $R$ upper triangular (same size as $A$).

\subsection*{Gram-Schmidt}

% https://www.math.uci.edu/~ttrogdon/105A/html/Lecture23.html

[Algorithms for classical and modified Gram-Schmidt (and construction by multiplication by R)]

[
Classical is numerically unstable because ...

Modified Gram-Schmidth -> Multiple vi summed "in parallel" (requires more space unless the input can be overwritten), on each iteration j the last term of vj is summed so it is normalized producing qj and used for the following iterations.

Relation to projection matrices:
\[
\bm v_j =
    \bm a_j - P_{\bm q_1} \bm a_j 
    - P_{\bm q_2} \bm a_2 -
    \cdots - P_{\bm q_{j-1}} \bm a_j
    =
    (I - \hat{Q}_{j-1} \hat{Q}_{j-1}\transp)\,\bm a_j
\]

$(I - \hat{Q}_{j-1} \hat{Q}_{j-1}\transp)$ is the complementary projection of $P_{\hat{Q}_{j-1}}$ onto 
]

\subsection*{Householder}

Main idea: build a reflection out of an orthogonal projection onto a hyperplane $U$ (perpendicular to a vector $\bm v$). Note that $U$ is a subspace of dimension  $n-1$.
    
An orthogonal projection `along' the vector $\bm v$, and `onto' the hyperplane $U$, has the matrix: \[ P_U = P_{\bm v^\perp} = I - \frac{\bm v\bm v\transp}{\bm v\transp\bm v} \]
    
    Now, we build a Householder reflection by adding twice the difference between the input vector $\bm x$ and projected vector $P_{\bm v^\perp}\bm x$:
    \[
        H_{\bm v}\bm x = \bm x + 2\, ( (P_{\bm v^\perp}\bm x) - \bm x)
        \implies
        H_{\bm v}
        =
        I + 2\, (P_{\bm v^\perp} - I)
        =
        \boxed{
            I - 2\, \frac{\bm v\bm v\transp}{\bm v\transp\bm v}
        }
    \]
A target vector $\hat{\bm a} =\pm\norm{\bm a}\bm e_1$ (with all zero coordinates, except the first one) can be reached by a suitable selection of the hyperplane $U$: in particular $\bm v=\bm a - \hat{\bm a}$ is chosen, so that $H_{\bm v} \bm a = \hat{\bm a}$.

[TODO: proof (this one is complicated...)]
    
The repeated application of this technique for each column of $A$, using one reduced dimension each time, results in an upper triangular matrix $R$.

Properties of any Householder reflection $H =H_{\bm v}$:
\begin{itemize}
    \item It is symmetric (/Hermitian): $H=H\transp$.
    \item It is orthogonal (/unitary): $H^{-1}=H\transp$; columns form an orthonormal basis.
    \item It is its own inverse: $H^{-1} = H$. Therefore we have: $H^2 = I$ (in group theory lingo, we say that $\mathrm{Order}(H)=2$).
    \item It has the eigenvalue  $-1$ with multiplicity $1$ and eigenvector $\bm v$ (eigenspace $\sspan(\bm v)$, with $\mu_g(-1) = \mu_a(-1) = 1$).
    \item It has the eigenvalue $+1$ with multiplicity $n-1$ and the eigenvectors are all vectors $\bm w$ in the hyperplane $U$; that is, any vector $\bm w$ such that $\bm w\perp \bm v$ (eigenspace $(\sspan(\bm v))^\perp$, with $\mu_g(1) = \mu_a(1) = n-1$).
\end{itemize}


[Householder (QR construction by multiplication by $Q$), 

Columns of $Q$ end up being a basis of $\range(A)$
]

\subsection*{Givens}

[Givens rotations magic...]

[Givens is, in general less efficient that Householder, but is still very useful for the case that the original matrix already had many zeros (sparse) at known locations.]

\addsec{Generalized inverses}

Generalize $A^{-1}$ for singular (square) matrices and rectangular ones.

Properties:

\begin{enumerate}
    \item $A A^\dagger A = A$
    \item $A^\dagger A A^\dagger = A^\dagger$
    \item $(A A^\dagger)\transp = A^\dagger A$
    \item $(A^\dagger A)\transp = A A^\dagger$
\end{enumerate}

[Projection using (1,4)]

[Minimal norm solution is not unique and least squares solution is not unique, but solution that is both minimal norm AND least squares IS unique.]

\subsection*{Moore-Penrose pseudoinverse}

It is the unique (1,2,3,4)-inverse.

From property (1):
\[
    P_{\range(A)} = AA^\dagger
\]
\[
    P_{\range(A)}^2 = P_{\range(A)} \quad \text{(idempotent, a valid projection matrix)}
\]
And also, using property (3):
\[
    P_{\range(A)} = P_{\range(A)}\transp \quad \text{(the projection is orthogonal)}
\]

If $A$ has $m\ge n$ and $A$ is full rank, then:
\begin{itemize}
    \item $A^\dagger = (A\transp A)^{-1}A\transp$

    (the inverse $(A\transp A)^{-1}$ exists because $A$ is full column rank)
    \item $A^\dagger A = I$ ($A^\dagger$ is left inverse)
    
    Proof: $A^\dagger A = (A\transp A)^{-1}A\transp A = I$
\end{itemize}

For SVD: $A=U\Sigma V\transp$, then
\[
    A^\dagger = V \Sigma^\dagger U\transp
\]

\begin{itemize}
    \item
    If $m\ge n$:
    \[\Sigma^\dagger = 
    \begin{pNiceMatrix}
    \hat\Sigma^\dagger & 0 
    \end{pNiceMatrix}
    \in \RR^{n\times m}
    \]
    \[\hat\Sigma^\dagger = 
    \diag(\sigma_1^\dagger, ..., \sigma_n^\dagger)
    \in \RR^{n\times n}
    \]
    \item
    If $m< n$:
    \[\Sigma^\dagger = 
    \begin{pNiceMatrix}
    \hat\Sigma^\dagger \\ 0 
    \end{pNiceMatrix}
    \in \RR^{n\times m}
    \]
    \[\hat\Sigma^\dagger = 
    \diag(\sigma_1^\dagger, ..., \sigma_m^\dagger)
    \in \RR^{m\times m}
    \]
\end{itemize}

Useful property!: If $\hat Q\in\RR^{m\times n}$ is \emph{semi}-orthogonal ($m>n$), then:
\[
    \boxed{\hat Q^\dagger = \hat Q\transp}
\]
(this is because $(Q\transp Q)^{-1} = (I)^{-1}= I$)

\addsec{Least Squares}
For the system of equations $A\bm x = \bm b$, the residual/error vector is defined as:
\[\bm r = \bm b - A\bm x\]
This system may have any number of solutions (one, zero or infinite).

Now, we impose $\bm r$ to be orthogonal (i.e.\@ \emph{normal}) to $\range(A)$ by forcing $\bm r\in \Ker(A\transp)$; then, we call ``solution'' $\bm x^*$ to a vector $\bm x$ that satisfies this property:
\[ A\transp \bm r =
A\transp (A\bm x^*-\bm b) = \bm 0
\implies
\boxed{A\transp A\bm x^* = A\transp \bm b}
\]
The equations of this system are called \emph{normal} equations (AFAIK, the term \emph{normal} here refers to orthogonality). Also, note that $A\transp A$ is a \emph{normal matrix} because it is symmetric.

Let's assume $A$ to be full column rank (implying also $m\ge n$), then $A\transp A$ is also full rank (and invertible). This also implies that $\bm x^*$ is unique and that it is an exact solution only when $A$ is square ($m=n$).

This can be expressed in terms of the Moore-Penrose pseudoinverse:
\[
\bm x^* = A^\dagger \bm b = (A\transp A)^{-1}A\transp \bm b
\]

Using the properties of $A^\dagger$, it can be shown that $\bm x^*$ is LS solution, meaning that:
\[
\bm x^* = \arg\min_{\bm x} \normtwo{\bm b - A\bm x}
\]
([TODO: check this claim] if $A$ was not full rank, there are possibly multiple LS solutions; using the properties of $A^\dagger$, it can be shown that $\bm x^*=A^\dagger \bm b$ is the one with minimal norm.)

Now, let's see how this is related to the orthogonal projection onto $\range (A)$, which is \[P=AA^{\dagger}.\]

% . But, in particular for $\normtwo{\bm r}$ minimized (i.e.\@ least squares solution) and then $\normtwo{\bm x^*}$ and .

% Use the orthogonal projection $P$ that maps ``into'' range of $A$ as follows:
% \[A\bm x^* = P\bm b \]

% Note that, by this choice $A\transp \bm r=\bm 0$, that is, $\bm r\perp \range(A)$ (use $A^*$ instead of $A\transp$ for the complex case).


We can show that the following system:
\[
P\bm b = A\bm x^*
\]
has the same solution $\bm x^* = A^\dagger\bm b$. Geometrically it makes intuitive sense that among all the projections onto $\range(A)$, the orthogonal projection is the one that achieves the minimum distance (residual).

Proof (assume $A$ full column rank and $m\ge n$):
\[
A\bm x^* = P\bm b
\implies
A\bm x^* = AA^\dagger \bm b
\implies
A^\dagger A\bm x^* = A^\dagger AA^\dagger \bm b
\]

Now, using $A^\dagger A=I$ and $A^\dagger AA^\dagger = A^\dagger$:
\[
A\bm x^* = P\bm b
\implies
\boxed{\bm x^* = A^\dagger \bm b}
\]

\subsection*{Methods to solve LS}
[Only the third one works if $A$ is not full rank]

\begin{itemize}
    \item Normal equations: $A\transp A=A\transp b$; but solving it directly is inefficient.
    
    Instead, use Cholesky $A\transp A = L L\transp$, with $L$ lower triangular. Solve two systems: one for $L$ lower triangular (back-subst) and next for $L\transp$ upper triangular (forward-subst).
    
    This requires for $A$ to be full rank; otherwise the solution is not unique (the Cholesky decomposition is also not unique). The pseudoinverse would not have the expression $A^\dagger = (A\transp A)^{-1}A\transp$ (in particular $(A\transp A)$ is not invertible).
    
    \item Reduced QR: use $A=\hat{Q}\hat{R}$ and algebra magic to obtain:
    \[A^\dagger = \cdots = \hat R^{-1}\hat Q\transp\]
    
    Then solve via back substitution.
    
    \item Reduced SVD: use SVD of $A$ and algebra magic to obtain:
    \[
    A =\hat U\hat\Sigma V\transp
    \implies
    A^\dagger = V\hat\Sigma^{-1}\hat U\transp\]
    
    Note: here we are using the ``thin'' version of the ``reduced'' SVD, meaning that $\hat\Sigma$ is square; also we assume that $A$ is \emph{thin} or square ($m\ge n$).
    
    Then solve using orthogonal transposes and diagonal inverse.

    In principle, this would only work for $A$ full rank, but, for this method, there is a straightforward generalization: instead of $\hat \Sigma^{-1} = \diag(\sigma_i^{-1})$, use the pseudoinverse of $\hat\Sigma$, defined as:
    \[
        \hat\Sigma^{\dagger} = \diag(\hat\sigma^{\dagger}_i)
        ,\quad
        \hat\sigma^{\dagger}_i =
        \begin{cases}
         \sigma_i^{-1}, & \text{ if } \sigma_i\neq 0\\
         0, & \text{ if } \sigma_i = 0
        \end{cases}
    \]

    
    

\end{itemize}

\newpage

\addsec{Problem conditioning}

Idea: a nonlinear ``problem'' $\bm f(\bm x^*)$, with $\bm f:\bm X\to \bm Y$ is well-conditioned when difference $\norm{\bm f(\bm x^*+\delta \bm x)-\bm f(\bm x^*)}_Y$ is small for small $\norm{\delta \bm x}_X$.

Define: $\delta \bm f = \bm f(\bm x^*+\delta \bm x)-\bm f(\bm x^*)$ (approximated by a Taylor expansion).

Note: norms $\norm{\cdot}_X$, $\norm{\cdot}_Y$ can be different and even the vector spaces of $\delta \bm x$ and $\delta\bm f$ may be different (e.g.\@ $\RR^n$, $\RR^m$).

Definition of absolute condition number (for a particular pair of norms):
\[
\hat{\kappa}(\bm x^*)
=
\lim_{\delta\to 0^+}\,\,\,
\sup_{\norm{\delta\bm x}_X\le\delta}
\frac{
\norm{\delta \bm f}_Y
}{
\norm{\delta \bm x}_X
}
\]

If $\bm f$ is a linear transformation ($\bm f(\bm x) = A\bm x$), then $\hat{\kappa}$ is the induced matrix norm (the inner expression does not depend on $\delta$). Reason: just notice that for $\delta\neq 0$ the supremum does not depend on $\delta$.

First order Taylor expansion: $
\delta \bm f = \bm f(\bm x^*+\delta \bm x)-\bm f(\bm x^*)
\approx \bm f^\prime(\bm x^*)\, \delta \bm x
$

If $f$ is differentiable, this results in the Jacobian of $f$:
\[
\hat{\kappa}(\bm x^*)
=
\norm{J(\bm x^*)}_{X,Y}
\]

Definition of relative condition number (possible when $\bm f(\bm x^*)\neq \bm 0$):
\[
\kappa(\bm x^*)
=\lim_{\delta\to 0}
\sup_{\norm{\delta \bm x}_X \le \delta}
\frac{\phantom{..}
\displaystyle
    \frac{\norm{\delta \bm f}_Y}{\norm{\bm f(\bm x^*)}_Y}
\phantom{..}
}{\phantom{..}
\displaystyle
    \frac{\norm{\delta \bm x}_X}{\norm{\bm x^*}_X}
\phantom{..}
}
=
\frac{\phantom{..}
    \hat\kappa(\bm x^*)
}{\phantom{..}
\displaystyle
    \frac{\norm{\bm f(\bm x^*)}_Y}{\norm{\bm x^*}_X}
\phantom{..}
}
\]

If $\bm f$ is differentiable, we obtain:
\[
{\kappa}(\bm x^*)
=
\frac{
    \norm{J(\bm x^*)}_{X,Y}
}{\displaystyle
\frac{\norm{\bm f(\bm x^*)}_X}{\norm{\bm x^*}_Y}}
\]

Rule of thumb: problem is ill conditioned when ${\kappa}(x^*) \ge 10^6$ (this is possibly compared to machine precision of 15 decimals for double precision).

Example: difference of two numbers is ill-conditioned if the are close. (Note: the Jacobian of a scalar function is just the gradient vector). $\kappa$ is easy to compute in the infinity norm.
\[f(x_1, x_2) = x_1 - x_2 \implies J = \left(
1, -1
\right) \]
\[
\kappa = \kappa_{\infty,\infty}
=
\frac{
    \norm{J}_{\infty,\infty}
}{\displaystyle
\frac{\norm{f(x_1, x_2)}_\infty}{\norm{(x_1, x_2)}_\infty}}
=
\frac{
    2
}{\displaystyle
\frac{|x_1 - x_2|}{\max{(x_1, x_2)}}}
\implies
\boxed{
    \kappa = \frac{2}{|x_1 - x_2|}\,\max(x_1, x_2)
}
\]

Example: polynomial root finding is very ill-conditioned (small changes in coeffs lead to big changes), in fact, for the 2nd order decomposition with 2 multiple roots is $\kappa = \infty$.

\subsection*{Matrix problems conditioning}

\subsubsection*{Problem: Matrix by vector product}

Compute relative $\kappa$ associated with the problem/operation $\bm f(\bm x) = A \bm x$, where we study the behaviour of perturbations on $\bm x$.

It can be easily obtained (Jacobian of linear function):
\[
\kappa = \kappa(\bm x) =  \frac{\norm{A}\,\norm{\bm x}}{\norm{A\bm x}}
\]

So, $\kappa(\bm x)$ is large when $\bm x$ is close to a vector of $\Ker(A)$. (Note that $\kappa$ is not defined when $A\bm x=\bm 0$).

Property for $A$ square invertible: \[\frac{\norm{\bm x}}{\norm{A\bm x}} \le \norm{A^{-1}} \]

(\,\! Proof: $
    \norm{\bm x} = \norm{A^{-1}\, A\, \bm x}
    \le \norm{A^{-1}}\, \norm{A\, \bm x}
$ \,\!)

Then, we can write: \[\kappa \le \norm{A}\norm{A^{-1}} \]

Or define $\alpha$:
\[\alpha =
\left.
    \frac{\norm{\bm x}}{\norm{A\bm x}}
        \middle /
    \norm{A^{-1}}
\right.    
\]
\[\kappa = \alpha \norm{A}\norm{A^{-1}} \]

Where $\alpha$ depends on $\bm x$ and the selected norm.

For the $2$-norm, it can be proven that equality ($\alpha=1$) is attained when $\bm x$ is proportional/collinear to a right singular vector ($\bm v_i$) corresponding to the smallest singular value ($\sigma_m$; must be nonzero as $A$ is invertible).

\subsubsection*{Problem: solving a linear system}

Assuming $A$ is square (system is \emph{exactly determined}) and invertible, consider the problem:
\[
    \bm f : \bm b \mapsto A^{-1}\bm b
\]
Observing the previous problem, this is exactly the same after swapping $A$ with $A^{-1}$ and $\bm x$ with $\bm b$; and we observe perturbations on $\bm b$.

Therefore:
\[
\kappa = \kappa(\bm b) =  \frac{\norm{A^{-1}}\,\norm{\bm b}}{\norm{A^{-1}\bm b}}
\]
And also:
\[\kappa \le \norm{A}\norm{A^{-1}} \]

Or define $\beta$:
\[\beta =
\left.
    \frac{\norm{\bm b}}{\norm{A^{-1}\bm b}}
        \middle /
    \norm{A}
\right.    
\]
\[\kappa = \beta \norm{A}\norm{A^{-1}} \]

Now, recall that $A^{-1}$ has the inverse singular values of $A$.

Then for the $2$-norm, it can be proven that equality ($\beta=1$) is attained when $\bm b$ is proportional/collinear to a left singular vector of $A$ ($\bm u_i$) corresponding to the largest singular value of $A$ ($\sigma_1$).

\subsubsection*{Problem: solving a linear system considering perturbations on $A$}

Consider the same problem as before, where now $\bm b$ is fixed and $A$ is the input:
\[
    \bm f : A \mapsto A^{-1}\bm b
\]

[TODO: finish this]

\subsubsection*{Matrix condition number}

We found a tight upper bound for $\kappa$ in the last 3 problems. This bound is important for the study of stability of matrix algorithms so we define it as precisely the ``condition number of a matrix $A$'' (not of a particular problem):
\[
    \boxed{\kappa(A) := \norm{A}\,\norm{A^{-1}}}
\]
Note that $\kappa(A)\ge 1$ for any induced matrix norm:
\[
    \kappa(A) = \norm{A}\,\norm{A^{-1}} \ge \norm{A\, A^{-1}} = \norm{I} = 1
\]
(Note: this is NOT true for general submultiplicative norms, e.g.\@ take $\norm{A}_{\alpha} = \normfrob{A}/n$, where $n$ is the size of $A$ and $ \normfrob{\cdot}$ is the Frob norm, then this is a valid norm and is submultiplicative, but $\norm{I}_{\alpha}=\sqrt{n}/n\le 1$)

For the 2-norm, it can be expressed as:
\[
\boxed{
\kappa_2(A)
    = \normtwo{A}\,\normtwo{A^{-1}}
    = \frac{\sigma_1}{\sigma_n} \ge 1
}
\]
In the geometrical interpretation of the SVD, $\kappa_2(A)$ could be seen as the output hyperellipse axial ratio.

For $A$ not invertible or rectangular, define:
\[\kappa(A) := \norm{A}\,\norm{A^{\dagger}}\]

\subsubsection*{Conditioning of matrix eigenvalue problems}
[TODO: check and rewrite all here]

Property: $A$ with eigv $\lambda$ and $A+\delta A$ with eigv $\lambda + \delta \lambda$.

Then: $|\delta \lambda| \le \normtwo{\delta A}$ with equality if $\delta A$ is multiple of identity matrix.

As a consequence: $\hat{\kappa} =1$, $\kappa= \normtwo{A}/|\lambda|$.

Matrix eigenvalue computation is ill-conditioned in general, but when it is symmetric, this becomes well-conditioned.

From a given polynomial we can find a ``companion matrix'' for which it is the characteristic polynomial.

%The usual case is $A\bm x=\bm b$, solve for $\bm x$ (output $f(x^*)$) given $A$ and $b$ (inputs $x^*$).

\subsubsection{Conditioning of Least squares methods}

...

[TODO: condition number etc.]

Normal equations, worse than SVD and QR.

% [Recall that the projection onto range $A$ is $A\, A^\dagger$, and for projection onto range $\hat Q$ orthogonal is $\hat Q\hat Q\transp$ (the stuff to invert is identity)]

%[TODO: proof that $\mathrm{range}(A) = \mathrm{kernel}(A\transp)^\perp$]

% [Projection of $\bm b$ into the range of $A$ and then solve the resulting system]

\newpage

\addsec{LU decomposition and Gaussian elimination}
The LU decomposition of a matrix $A$ is
\[A = LU\]
with $L$ lower triangular with ones in diagonal (a.k.a.\@ lower unitriangular), $U$ upper triangular.

This variant is called Doolittle decomposition (in the Crout decomposition, $U$ is upper unitriangular and $L$ is lower triangular with arbitrary diagonal).

It is obtained via Gaussian elimination (sequence of row transformations).
\[
A^{(0)} = A,\quad
A^{(k)} = L_k A^{(k-1)},\quad
A^{(n-1)} = U
\]
\[
\implies A^{(k)} = L_{k}\cdots L_{2}L_{1}\, A
\]
\[
\implies U = A^{(n-1)} = L_{n-1}\cdots L_{2}L_{1}\, A
\]
\setcounter{MaxMatrixCols}{20}
\[
L_k =
\begin{pNiceMatrix}
1 & \\
&\Ddots \\
\\
&&1 & \\
&&-\ell_{k+1,k}&  \\
&&\Vdots & \\
&&-\ell_{s,k}&  \\
% &&\Vdots & \\
&&\Vdots &&&&&\\
&&-\ell_{n,k}&&&&&& 1\\
\end{pNiceMatrix}
\]
Where:
\[
-\ell_{s,k} =
-\frac{a_{sk}^{(k-1)}}{a^{(k-1)}_{kk}}
,\quad
k \in \{1, ..., n-1\}
,\quad
s \in \{k+1, ..., n\}
\]
Note that, for this to be feasible, the denominators  ${a^{(k-1)}_{kk}}$ must be nonzero. To begin with, $a_{11}$ must be nonzero; in fact, the denominators [TODO: add reference] are precisely the $n-1$ first ``leading minors'' (a.k.a.\@ ``principal minors'', those are the determinants of all the upper left square submatrices of $A$); this does NOT include the last determinant, $\det(A)$ itself, so this process may work for some singular matrices $A$.

Define the vector:
\[
\bm \ell_k = (\underbrace{0,...,0}_{k}, \ell_{k+1,k}, ..., \ell_{s,k}, ..., \ell_{n,k})
\]
(the $k$-th position contains the last zero before $\ell_{k+1,k}$).

Then:
\[
L_k = I - \bm \ell_k \, \bm e_k\transp
\]
The inverse transformation has matrix (also lower triangular):
\[
L_k^{-1} = I + \bm \ell_k \, \bm e_k\transp
\]

Proof:
\[
(I - \bm \ell_k \, \bm e_k\transp)
(I + \bm \ell_k \, \bm e_k\transp)
=
I
+
\bm \ell_k \, \bm e_k\transp
\,(
I
-
I+
\bm \ell_k \, \bm e_k\transp)
=
I+
\bm \ell_k \, \underbrace{
    \bm e_k\transp \bm \ell_k}_{0} \, \bm e_k\transp
= I
\]

Now, sequences of transformations in order do not modify $\bm \ell_k$:
\[
L_k^{-1} \, L_{k+1}^{-1} =
I
 + \bm \ell_k \bm e_k\transp
 + \bm \ell_{k+1} \bm e_{k+1}\transp
 + \bm \ell_{k}
 \underbrace{\bm e_{k}\transp\bm \ell_{k+1}}_{0} \bm e_{k+1}\transp 
=
I
 + \bm \ell_k \bm e_k\transp
 + \bm \ell_{k+1} \bm e_{k+1}\transp
\]

Therefore:
\[
L=
L_1^{-1} L_2^{-1}...
L_{n-1}^{-1}
=
I +
\sum_{k=1}^{n-1} \bm \ell_k \bm e_k\transp
=
\begin{pNiceMatrix}
1 \\
&1 \\
\bm \ell_1 & &\Ddots \\
& \bm \ell_2&& 1\\
 &{~}&\Cdots&  \bm\ell_{n-1} & 1
\end{pNiceMatrix}
\]

Warning on confusing sign convention: $L_k$ contains $-\bm\ell_k$ and $L_k^{-1}$ contains $+\bm\ell_k$, and $L$ is built using $L_k^{-1}$.

\subsection*{Existence of LU decomposition}

As mentioned before, this method only works when the $n-1$ ``leading minors'' are nonzero.

But note that those are not \emph{all} the matrices that admit LU decomposition; for example, the zeros matrix can be decomposed with $U=0$ ($L$ can be chosen as \emph{any} lower unitriangular matrix), but there is always a zero in the denominator of the algorithm (even with pivoting).

Nevertheless we have the following:
\begin{itemize}
    \item In the case that $A$ is full rank, $A$ admits LU factorization iff the ``principal minors'' are nonzero (now we can include the determinant of $A$, as $A$ is full rank). This is good news, as this is the only type of matrices that this method works on.
    
    In such case the described algorithm produces the unique decomposition.

    \item If the rank of $A$ is $n-1$ this method still works, but the decomposition may not be unique (the method only produces only one of those decompositions).

    \item In general, if $A$ is not full rank, the condition is more complicated.
    
    Whole story here: \url{https://arxiv.org/pdf/math/0506382.pdf}.
\end{itemize}

[TODO: mention gaussian elimination and complete system solving.
Mention also backsubstitution. Explain memory savings using a single matrix.]

\subsection*{Pivoted LU decomposition}

To avoid the case where the $LU$ decomposition does not exist for an otherwise perfectly common invertible matrix (and a corresponding solvable system), we can permute the rows of the matrix; then the $LU$ decomposition is always feasible for regular $A$ (in fact, this should work for all square matrices $A$ with $\rank(A) \ge n-1$).

Not only this makes the problem feasible in theory, but also, because of the choices available when selecting the pivots, it can be made numerically more stable.

Recall that, for a permutation matrix $P$: pre-multiplication permutes the rows ($PA$) and post-multiplication permutes the columns ($AP$).

There are two pivoting methods: partial pivoting and full pivoting.

\subsubsection*{Partial pivoting}

Partial (row) pivoting can be seen first as an row permutation applied before each column elimination, resulting in:
    \[L_{n-1}P_{n-1}...L_{1}P_{1} A = U\]

Each $P_k$ is very simple: only two rows below the diagonal are exchanged.
    
The optimal row choice (for stability) is to leave the largest element ``alive'' and annihilate the rest.
    
This can be rewritten as:
\[
    L_{n-1}P_{n-1} \cdots L_{1}P_{1} =
    L_{n-1}^\prime \cdots L_1^\prime \, P_{n-1} \cdots P_{1}
\]
    
Where: $L_{n-1}^{\prime} = L_{n-1}$, $L_{n-2}^{\prime} = P_{n-1} L_{n-1}P_{n-2}^{-1}$, ..., $L_1^\prime = P_{n-1}...P_1 L_1 P_1^{-1} P_{n-1}^{-1}$.
    
In general:
\[L^\prime_k =
    P_{n-1}P_{n-2}\cdots P_{k+1}\,
    L_k\,
    P_{k+1}^{-1}\cdots P_{n-1}^{-1}\,
\]
    
Now, for the $\alpha$-th column, observe the following:
\[
    L_\alpha^\prime = P_\beta L_\alpha P_\beta^{-1}
    , \text{ with } P_{\beta} = \prod_{\beta_i>\alpha} P_{\beta_i}
    \,.
\]

The transformation done on $L_\alpha$ is to ``unpermute'' ($P_\beta^{-1}$) columns that are to the left of $\alpha$ (this does not affect the elements of $\bm\ell_\alpha$), and permute ($P_\beta$, with same pattern) rows of $L_\alpha$ below $\alpha$ (this only reorders elements of $\bm\ell_\alpha$ into $\bm\ell_\alpha^\prime$).

Then, aside from reordering $\bm\ell_\alpha$, this has exchanged rows and then columns (in the reverse order) in the diagonal: as the latter changes leave the all-ones diagonal unaltered, the resulting matrix is $L^\prime_\alpha = I - \ell_k^\prime\bm e_k\transp$. So, the defining property of is $L_\alpha$ (i.e.\@ identity plus elements in $\alpha$-th column under diagonal) is preserved in $L^\prime_\alpha$.

This argument can be summarized like this:
\[
    L^\prime_\alpha = 
    P_\beta L_\alpha P_\beta^{-1}
    =
    P_\beta (I - \ell_k\bm e_k\transp) P_\beta^{-1}
    =
    P_\beta P_\beta^{-1} - P_\beta\ell_k\bm e_k\transp P_\beta^{-1}
    =
    I - \ell_k^\prime\bm e_k\transp
\]

Therefore, the decomposition
\[
    L_{n-1}P_{n-1} \cdots L_{1}P_{1} =
    \underbrace{L_{n-1}^\prime \cdots L_1^\prime}_{L^{-1}}
    \,
    \underbrace{P_{n-1} \cdots P_{1}}_{P}
\]
is a product of a lower unitriangular matrix $L^{-1}$ (with lower part defined by the vectors $-\bm\ell^\prime_k$) and a permutation matrix $P$.

Then, we have:
\[
    L^{-1} P A = U
    \implies 
    \boxed{
        P A = L \,U
    }
\]

With \[L = (L_{1}^\prime)^{-1} \cdots (L_{n-1}^\prime)^{-1}
=
    \begin{pNiceMatrix}
    1 \\
    &1 \\
    \bm \ell_1^\prime & &\Ddots \\
    & \bm \ell_2^\prime&& 1\\
     &{~}&\Cdots&  \bm\ell_{n-1}^\prime & 1
    \end{pNiceMatrix}
.\]

Implementation is similar as with no pivoting, but with some row/column permutations of the intermediate results, wherever it makes sense. The resulting permutation matrix $P$ is stored as a vector of $n$ indices.

Existence of the decomposition $PA = LU$ is guaranteed for all $A$ square (although this method only works for \emph{some} matrices; but at least it works for all invertible matrices $A$ and also for $\rank(A)=m-1$).

A similar method would be partial column pivoting, that uses column permutations instead of row permutations: we get $AP = LU$.

Note: in some sources this decomposition is written as $A=P\transp LU$ (and in the case of column pivoting, $A=LUP\transp$). Note that a permutation matrix $P$ is orthogonal $P^{-1}=P\transp$.

\subsubsection*{Complete pivoting}

Introduces column pivoting matrices $Q_k$.
\[L_{n-1}P_{n-1}...L_{1}P_{1}\, A\, Q_1 Q_2 ...Q_{n-1} = U
\implies
    \underbrace{L_{n-1}^\prime \cdots L_1^\prime}_{L^{-1}}
    \,
    \underbrace{P_{n-1} \cdots P_{1}}_{P}
    \,
    A
    \,
    \underbrace{Q_{1}Q_{2} \cdots Q_{n-1}}_{Q}
    =
    U
\]
\[
\implies
\boxed{
    P A\, Q = L \,U
}
\]

Not very used in practice, because of time-consuming pivot selection, in exchange for only marginal stability improvement.

\subsection*{\texorpdfstring{$LDL\transp$}{LDLT} decomposition}

For symmetric matrices for which the LU decomposition exists (consider without pivoting, for simplicity), the diagonals of $U$ in the LU decomposition can be factored out (if some are zero, there are degrees of freedom to choose some values [check this?]), to build the decomposition:
\[
    A = L D L\transp
\]

\subsection*{Cholesky decomposition}

Taking the $LDL\transp$ decomposition of a positive semidef.\@ matrix $A$, the diagonal of $D$ has nonnegative elements, and we can write:
\[
A = L^\prime D L^\prime{}\transp 
    =
\underbrace{L^\prime D^{1/2}}_L 
\underbrace{D^{1/2} L^\prime{}\transp}_{L\transp}
=
L \,L\transp
\]

If $A$ is pos.\@ def.\@ (strictly), then this decomposition is unique and the diagonals of $L$ are nonzero.

If it is only pos.\@ semidef.\@ this decomposition also exists but it is never unique for $n>1$ [TODO: check this].

Note that this decomposition always exists for positive semidef.\@ matrices.

Also, for any (pos.\@ semidef.) matrix, the $LDL\transp$ decomposition can be directly obtained from a Cholesky decomposition and vice~versa.

\subsection*{Shur form}
Shur form/decomposition (exists for any matrix, not unique):
\[
    A = Q T Q\transp, \quad
    \text{ with $Q$ orthogonal/unitary and $T$ upper triangular.}
\]
It gives an orthogonal similarity transformation, it is a generalization of the orthogonal diagonalization: $A = Q \Lambda  Q\transp$, which exists for normal matrices and, in particular, in the case of Hermitian/real symmetric matrices, the eigenvalues $\lambda_i$ are guaranteed to be real.

\newpage
\addsec{Eigenvalue problems}

General assumption: $A$ is Hermitian (incl.\@ real symmetric). So, all eigenvalues $\lambda_i$ are real and it is always orthogonally diagonalizable (because Hermitian matrices are normal).


[TODO Add pagerank]

\subsection*{General structure of the eigenvalue algorithms}

$A$ is symmetric, then $Q\transp A Q = \Lambda$ diagonal exists (in general $\Lambda$ would be triangular).

But, because of fundamental issues (Abel's impossibility theorem), this cannot be done in a finite number of steps, as there is not an ``algebraic solution'' in general. All algorithms are iterative by necessity.

The idea of the iterative algorithms is to build a sequence of orthogonal matrices $Q_i$ such that:
\[
Q^*_n Q^*_{n-1}...Q^*_1Q^*_2
\, A\,
Q_1 Q_1...Q_{n-1}Q_n \xrightarrow{\,(n\to\infty)\,} \Lambda
\]
Ideally, the convergence.

The same algorithms used for general matrices are used for symmetric ones, with some simplifications where suitable.

We can classify the methods as [TODO: rethink this]:
\begin{enumerate}
    \item Simple algorithms that produce eigenvector estimates: power iteration, inverse power iteration, inverse shifted power iteration.
    
    The corresponding eigenvalue is then obtained via the Rayleigh quotient.
    
    \item Rayleigh quotient iteration: combines simultaneously eigenvector and eigenvalue approximation (very fast).
    
    \item Practical algorithms: QR iteration (``pure'': wihout shifts; and ``practical'': with shifts), Arnoldi iteration method (used for sparse matrices).
\end{enumerate}


\subsection*{Companion matrix}

[TODO: ...]


\subsection*{Gershgorin circles}

Gershgorin Theorem I: if $\lambda$ is eigenvalue of $A$, then there exists an index $i$ (corresponding to the $i$-th row of $A$), such that:
\[
    |\lambda - A_{ii}| \le
    \sum_{j=1,\,j\neq i}^n |A_{ij}|
\]

In other words: every single eigenvalue must lie in at least one of the Gerschgorin disks (including the boundaries).

Define the Gerschgorin disk associated with the $i$-th row of $A$ as:
\[
    D_i = \{z\in \CC:\,\, |z-A_{ii}|\le d_i\},
    \text{ with } d_i = \sum_{j=1,\,j\neq i}^n |A_{ij}|
\]
That is: $D_i$ is a circle in $\CC$ centered at $A_{ii}$ with radius $d_i$.

This also applies if we used the columns of $A$ instead of the rows (proven by using $A\transp$, which has the same eigenvalues as $A$ (same char-poly)).

Gershgorin Theorem II: if the union $G_1$ of $k$ disks is disjoint from the union of the other $n-k$ disks, $G_2$, then there are exactly 
(note: we can impose that the $k$ disks are non-concentric so that  eigenvalues are not allowed to be repeated).

In particular: 
the disjoint sets containing the unions of overlapping disks (non-concentric, meaning that $A_{ii}$ are all distinct), always contain exactly a number of eigenvalues equal to the number of such disks.

\subsection*{Rayleigh quotient}
Rayleigh quotient:
\[
    r(\bm x) = 
    \frac{
        \bm x\transp A \bm x
    }{
        \bm x\transp \bm x
    }
\]

It gives an ``aproximation'' of an eigenvalue in the direction of $\bm x$.

In the case that $\bm x = \bm q_i$ is an eigenvector (for $\lambda_i$):
\[
    r(\bm q_i) =
    \frac{
        \bm q_i\transp A \bm q_i
    }{
        \bm q_i\transp \bm q_i
    }
    =
    \lambda_i
    \frac{
        \bm q_i\transp \bm q_i
    }{
        \bm q_i\transp \bm q_i
    }
    =
    \lambda_i
\]

It can be proven that:
\[
    r(\bm x) = \arg\min_\alpha \norm{A\bm x - \alpha \bm x}
\]

[todo proof]

And also:
\[
    \nabla r(\bm x) = 
    \frac{2}{
        \bm x\transp \bm x
    }\,
    (A\bm x - r(\bm x)\, \bm x)
\]

Then, any eigenvector $\bm x = \bm q_i$ is stationary point: $\nabla r(\bm q_i)=\bm 0$.

Using Taylor expansion around $\bm q_i$: 
\[
    r(\bm x) - r(\bm q_i) = \OO(\normtwo{\bm x - \bm q_i}^2)
\]
Then, the eigenvalue estimation is \emph{quadratically accurate}.

Alternative expression with $\bm x$ as linear combination of eigenvalues (note that this is always possible because $A$ is symmetric, hence diagonalizable):
\[
    r(\bm x) =
    \frac{
    \displaystyle
        \sum_i^n \alpha_i \bm q_i
    }{
    \displaystyle
        \sum_i^n \alpha_i^2
    },
    \quad \text{where }
    \bm x = \sum_i \alpha_i \bm q_i
    \quad
    \text{ ($\alpha_i$ are unique coordinates of $\bm x$)}
\]


\subsection*{Power iteration}

This is the simplest iterative method. It produces a sequence of eigenvector estimates $\bm v^{(k)}$ that should converge to the one corresponding to the largest eigenvalue (if it is unique in abs.\@ value).

[todo two norm not needed, but useful for rayl quitient]

Seed: $\bm v^{(0)}, \, \text{ with }\normtwo{v^{(0)}} = 1$

At $k$-th iteration:
\[
    \bm w := A \bm v^{(k-1)}
\]
\[
    \bm v^{(k)} := \frac{\bm w}{\normtwo{\bm w}}
\]

This is called power iteration because at the $k$-th iteration we have a vector $\bm v^{(k)}$ proportional to $A^k \bm v^{(0)}$.

The eigenvalue estimate at the $k$-th iteration is given by the Rayleigh quotient:
\[
    \lambda^{(k)} := r(\bm v^{(k)}) = (\bm v^{(k)})\transp A \bm v^{(k)}
\]
(note that $\bm v^{(k)}$ is already normalized)

To analyze convergence we write $\bm v^{(k)}$ as:
\[
    \bm v^{(k)} = c_k \lambda_1^k
    \left(
        \alpha_1 \bm q_1
        + \sum_{i=2}^n
        \alpha_i
        \left(\frac{\lambda_i}{\lambda_1}
        \right)^i
        \bm q_i
    \right)
\]

For some constants $c_k$ (should be $\OO(1)$); $\alpha_i$ are the coordinates in the eigenbasis $\{\bm q_i\}_i$ of $\bm v^{(0)}$ (this is always possible when $A$ is diagonalizable).

This method converges to the largest (in $|\cdot|$) eigenvalue $\lambda_1$ when $|\lambda_1| > |\lambda_2|$ and at at rate $\OO(|\lambda_1 / \lambda_2|)$ per iteration in eigenvectors; $\OO(|\lambda_1 / \lambda_2|^k)$ per $k$ iterations and the square $\OO(|\lambda_1 / \lambda_2|^{2k})$ for eigenvalues.

Note though, in practice, it is rare that a matrix has exactly the same two equal eigenvalues at the numerical precision.

This is possibly the worst method (but is very simple and usually works).

\subsection*{Inverse power iteration}

This method modifies the power iteration to give the smallest eigenvalue.

Seed: $\bm v^{(0)}, \, \text{ with }\normtwo{v^{(0)}} = 1$

At $k$-th iteration:
\[
    \bm w := A^{-1} \bm v^{(k-1)} \qquad 
    (\text{in practice, solve the system: }A\bm w = \bm v^{(k-1)})
\]
\[
    \bm v^{(k)} := \frac{\bm w}{\normtwo{\bm w}}
\]
Eigenvalue estimates given by $r(\bm v^{(k)})
$.

This method is ``simple'' but is still very useful in practice when the target eigenvalue is already known.

\subsection*{Inverse shifted power iteration}

Main idea: modify the power iteration so that the eigenvalues are separated.

$A-\mu I$ shifts the eigenvalues to $\lambda_i-\mu$ (the eigenvectors are the same!), then if $\mu$ is close to a particular eigenvalue $\lambda_*$ of $A$, the matrix $(A-\mu I)^{-1}$ has a large eigenvalue $(\lambda_*-\mu)^{-1}$ (much larger than the second closest eigenvalue).

Seed: $\bm v^{(0)}, \, \text{ with }\normtwo{v^{(0)}} = 1$

At $k$-th iteration:
\[
    \bm w := (A-\mu I)^{-1} \bm v^{(k-1)} \qquad 
    (\text{in practice, solve the system: }(A-\mu I)\bm w = \bm v^{(k-1)})
\]
\[
    \bm v^{(k)} := \frac{\bm w}{\normtwo{\bm w}}
\]
Eigenvalue estimates also given by $r(\bm v^{(k)})
$.

Even if it would appear that the choice of $\mu$ could make $(A-\mu I)$ very ill-conditioned, in practice this is not actually a problem:  the matrix $(A-\lambda_i I)$ is not invertible, but the system $(A-\lambda_i I)\bm v = \bm 0$ has infinite solutions. So, with a suitable solver a nonzero solution can be found here [TODO: is this true?].

Convergence also requires the closest eigenvalue (in abs.\@ val.) to be unique (less likely if $\mu$ is ``random'').


\subsection*{Rayleigh quotient iteration}

Given that the Rayleigh quotient is an approximation of $\lambda_i$ we can use it as $\mu$ (updated at each iteration) in the inverse power iteration to accelerate it. 
Seed: $\bm v^{(0)},\,\lambda_0$.

At $k$-th iteration:
\[
    \mu := \lambda_{k-1}
\]\[
    \bm w := (A-\mu I)^{-1} \bm v^{(k-1)} \qquad 
    (\text{in practice, solve the system: }(A-\mu I)\bm w = \bm v^{(k-1)})
\]
\[
    \bm v^{(k)} := \frac{\bm w}{\normtwo{\bm w}}
\]
\[
    \lambda^{(k)} := r(\bm v^{(k)}) = (\bm v^{(k)})\transp A \bm v^{(k)}
\]

It is actually very fast (cubic convergence for eigenvalues).

\subsection*{Reduction to Hessenberg/triangular form}

A Hessenberg matrix is an almost-upper triangular matrix (lower, depending on the book) with one subdiagonal.


If a matrix is Hermitian/real symmetric and Hessenberg then it is tridiagonal (therefore, also sparse).

The efficient eigenvalue algorithms generally used in practice have two phases:
\begin{enumerate}
    \item Convert $A$ into Hessenberg form $H$ (tridiagonal, for $A$ symmetric).
    
    Can be done by a direct method (i.e. not iterative). The Arnoldi iterative method is suitable here when $A$ is sparse.
    
    \item Algorithm that converts $H$ into diagonal. This is generally done by an iterative method.
    
    For the Hermitian case, if the eigenvectors are not required, the second phase is faster than the first one.
\end{enumerate}

[TODO: finish this...]

[Important remark: $A=QHQ\transp$ is a similarity transformation, so it preserves eigenvalues (easily proven because the charPoly of $A$ is the same as the charPoly of $H$ using properties of the determinant)]

[Also: property: if $A$ is symmetric, $H = PAP\transp$ is also symmetric regardless of $P$ (!); but here $P=Q$ is also orthogonal, so it also preserves eigenvalues.]

[First phase: direct method.]

[Householder reflections]

[Directly getting the upper triangular is not possible because we need to pre- and post- multiply by $Q_i$].

\subsection*{QR iteration}
This is a very used algorithm in practice. It is an eigenvalue algorithm that relies on the QR decomposition (using Householder, Givens, etc.).

[``Simple'' and ``Practical'' (with shifts and deflations) versions]

[...]

\newpage
\addsec{Iterative methods for sparse matrices}

General assumption: $A\in\RR^{m\times m}$, with $m\gg 1$ and $A$ is sparse.

[TODO: definition of sparse: nnz$(A)=\OO(m??)$]

Goal: solve $A\bm x =\bm b$.

Then, we can compute $A\bm x$ efficiently: ``MatVec'', i.e.\@ matrix-vector product, is $\OO(m)$ (for general matrices we would need $\OO(m^2)$).

Some examples of sparse matrices: 
\begin{itemize}
    \item Banded matrices (incl.\@ diagonal matrix, tridiagonal, etc.), with $\OO(1)$ bands.
    
    This kind typically appears in the discretization of PDEs (finite element methods), because of the locality of derivatives.
    
    \item Permutation matrices.

    \item Hyperlink matrices (block diagonal with ``small'' blocks).
    
\end{itemize}

For general matrices, solving $A\bm x = \bm b$ with direct methods requires $\sim\!\OO(m^3)$. We may simplify this with the knowledge that $A$ is sparse (\emph{sparse direct methods}), but we will study here only iterative methods.

These iterative methods rely only on matrix-vector products (``MatVec'') as the most expensive operation. And in particular, one iteration uses only one MatVec.

The cost per iteration is then $\OO(m)$. If we could find a method where the number of iterations is $\OO(1)$ (typically 20-100(?)), then the total cost would be also linear.

Now, the product $A^k\bm b$ is also computed with MatVecs and we can generalize this to the multiplication by an $n$-th degree matrix polynomial ($p(A)\bm b$), where:
\[
    p(A) = a_n A^n +  a_{n-1} A^{n-1} + \cdots + a_1 A + a_0 I = \sum_{i=0}^n a_i A^i
\]

This also has the utility that, if $A$ is similar to a ``simpler''  matrix $B$ (e.g.\@ diagonalization, Shur form), $A=SBS^{-1}$, then:
\[
    p(A) = S\, p(B)\, S^{-1} \qquad \text{(because $A^{i}=SBS^{-1}SBS^{-1}\cdots SBS^{-1}=SB^{i}S^{-1}$)}
\]

In the particular case of diagonalization ($A=X\Lambda X^{-1}$):
\[
    p(A) = X\, p(\Lambda)\, X^{-1}; \qquad \text{and }p(\Lambda)
        = \diag(p(\lambda_1),\, p(\lambda_2),\, ...,\, p(\lambda_m))
\]

% \subsection*{Characteristic polynomial}

We denote $\chi_A(\lambda)$ as the characteristic polynomial of $A$:
\[\chi_A(\lambda) = \det(\lambda I - A) = \sum_{i=0}^{m}a_i\,\lambda^i \]
Note that $\chi_A$ is monic by definition, and therefore $a_n=1$.

Also, $a_0 = \det(-A)$.

We know that, by the Cayley-Hamilton theorem, $A$ is root of its the matrix-valued characteristic polynomial $\chi_A(X)$:
\[\chi_A(X) = \sum_{i=0}X^{i}a_i
    \implies
    \boxed{ \chi_A(A) = 0 }
\]

If $A$ is invertible, $a_0=\det(-A)\neq 0$, then we can express $A^{-1}$ as:
\[
    A^{-1} = -\frac{1}{a_0}A^{m-1} - \frac{a_{m-1}}{a_0}A^{m-2} - \cdots - \frac{a_1}{a_0} I
    = \boxed{-\sum_{i=0}^{m-1} \frac{a_{i+1}}{a_0} A^{i}}
\]
That is, $A^{-1}$ is expressed as a polynomial of degree $m-1$ (or less) in $A$.

[Of course, this has little practical value for the numerical computation of $A^{-1}$: this requires the characteristic polynomial coefficients and $m-1$ matrix products.]

Main idea: try to find a low-degree polynomial $p_k(X)$, such that $p_k(A)\bm b\approx A^{-1}\bm b$, for some notion of approximation ($\approx$). That is:
\[
    p_k(A)\bm b=
     \left(-\sum_{i=0}^{k} \hat{a}_i A^i\right)\bm b
        \approx
    \left(-\sum_{i=0}^{m-1} \frac{a_{i+1}}{a_0} A^{i}\right)\bm b = A^{-1} \bm b
\]
Ideally with: $k \ll m$.

Note that we don't really need that $p_k(A)\approx A^{-1}$, but only that $p_k(A)\bm b\approx A^{-1}\bm b$, which is a less strict condition.

\subsection*{Krylov subspaces}

Given a square matrix $A\in\RR^{m\times m}$ and a vector $\bm b \in \RR^m$ we define their $n$-th order Krylov subspace as:
\[
    \Kry_n(A,\, \bm b) = \sspan(\bm b,\, 
    A\bm b,\,
    A^2\bm b,\,
    ...\,,\,
    A^{n-1}\bm b
    )
    \subseteq \RR^m
\]
[notation: $m$ is the size/order of the matrix $A$ and $n$ is the order of the subspace]

We write $\Kry_n = \Kry_n(A,\, \bm b)$ when the pair $(A,\bm b)$ is obvious.

$\bm b$ can be written as $A^0 \bm b$, and also $\Kry_1 = \sspan(\bm b) = \sspan(A^0\bm b)$.

Note that, because we are adding one more vector to the span:
\[
    \Kry_1 \subseteq \Kry_2 
        \subseteq \Kry_3
        \subseteq \cdots
        \subseteq \Kry_{n-1}
        \subseteq \Kry_{n}
        \subseteq \RR^m
\]

Ideally, we would get that, for $n=m$, $\Kry_{m} = \RR^m$, but this is not guaranteed in general. This requires that for all $n\le m$, $\dim \Kry_{n} = n$.

(This is not necessarily problematic in practice.)

Now, any vector $\bm x$ in $\Kry_n$ can be written as (uniquely, when $\dim(\Kry_n)=n$):
\[
    \bm x \in \Kry_n \implies
        \bm x = \sum_{j=0}^{n-1}\gamma_j A^j \,  \bm b
    =
        \underbrace{
            \left(\sum_{j=0}^{n-1}\gamma_j A^j \right)
        }_{p_{n-1}(A)}
        \,\bm b
    =
        {p_{n-1}(A)}
        \,\bm b
\]
Where ${p_{n-1}(X)}$ is some matrix polynomial of degree $n-1$.

If $\dim(\Kry_n) = n$, then we can always find an orthogonal basis with $n$ vectors $\{\bm q_i\}_{1\le i\le n}$ of the subspace $\dim(\Kry_n)$:
\[
    \Kry_n
        = \sspan(\bm b,\, A\bm b,\, A^2\bm b, \dots ,\, A^{n-1} \bm b)
        = \sspan(\bm q_1,\,\bm q_2,\,\bm q_3, \dots ,\,\bm q_n)
\]


\subsection*{Arnoldi iteration}

The Arnoldi iteration algorithm is used to transform an arbitrary matrix $A$ into a similar Hessenberg form: $Q\transp AQ=H$. It is an iterative method, so we usually stop at an iteration, say $n$, and then we have: $Q_n\transp A Q_n\transp = H_n$.

It is shown later that this algorithm is closely related to Krylov spaces.

We studied direct methods that produce an orthonormal basis (orthogonal matrices): Gram-Schmidt (original or modified), Householder and Givens. Of these, only Gram-Schmidt outputs a vector a each step, but it is only useful in QR factorization ($A=QR$); the others can be used for computing a Hessenberg form similar to $A$ ($A=Q\transp HQ$), but require to completely finish to get a useful output; as $m$ is very large, this would take too long.

The Arnoldi iteration is similar to Gram-Schmidt in this respect: it produces a new vector $\bm q_n$ per iteration, which is appended to an orthonormal basis (forming a matrix $Q_n$); it also produces a sequence of Hessenberg matrices $H_n$ of increasing size ($H_n$ contains the elements of $H_{n-1}$). Under ideal conditions, we would arrive at at the $m$-th iteration with $Q=Q_m$ and $H=H_m$, such that $A =QHQ\transp$.

Important remark: Arnoldi iteration is only efficient for the case of $A$ sparse (it leverages the ``efficiency'' of MatVecs), otherwise, it is more efficient to use the Householder method (or Givens, depending on the case).

Recall that the Hessenberg form transformation produces the decomposition:
\[
    A = Q H Q\transp \iff A Q = QH
\]
With $Q$ an orthonormal basis of $\range(A)$. $A$ and $Q$ are of size $m\times m$.

These matrix products can be rewritten in terms of ``MatVecs'' like this:
\[
A Q = 
\begin{pNiceMatrix}[vlines]
    A\, \bm q_1 &
    \cdots &
    A\, \bm q_{m-1} &
    A\, \bm q_m
\end{pNiceMatrix}
=
\begin{pNiceMatrix}[vlines]
    Q\, \bm h_1
    &  \cdots &
    Q\, \bm h_{m-1} &
    Q\, \bm h_m
\end{pNiceMatrix}
=
QH
\]
If we remove the last column on both sides we get:
\[
\begin{pNiceMatrix}[vlines]
    A\, \bm q_1
    &  \cdots &
    A\, \bm q_{m-2} &
    A\, \bm q_{m-1}
\end{pNiceMatrix}
=
\begin{pNiceMatrix}[vlines]
    Q\, \bm h_1
    &  \cdots &
    Q\, \bm h_{m-2} &
    Q\, \bm h_{m-1}
\end{pNiceMatrix}
\]
Now, note that the LHS is the matrix $A$ multiplied by the matrix $Q$ with one less column, we call this truncated matrix $Q_{m-1}\in\RR^{m\times (m-1)}$. Denote also $Q_m=Q$.

On the RHS we have a similar situation, but now note that the products
    \[  Q_m\,\bm h_{i}  \quad \text{for $i\in \{1,\, ...\,,\, m-2\}$}\]
ignore the last column of $Q$ because, by the Hessenberg property of $H$, the last element of  $\bm h_{i}$ is zero.% This is not the same for $i=m-1$, so we treat it separately.

Then, consider the vectors $\hathat{\bm h}_i$ (size $m-1$), that result from dropping the last element of $\bm h_i$. These vectors form the columns of $\tilde{H}_{m-2} \in\RR^{(m-1)\times (m-2)}$, with both the last row and last two columns removed from $H$, then, we can write the last equation like this:
\[
    \begin{pNiceMatrix}[vlines]
        A\, \bm q_1
        & \cdots &
        A\, \bm q_{m-1}
    \end{pNiceMatrix}
    =
    \begin{pNiceMatrix}[vlines]
        Q_{m-1}\, \hathat{\bm h}_1
        &  \cdots &
        Q_{m-1}\, \hathat{\bm h}_{m-2} &
        Q_m \, {\bm h}_{m-1}
    \end{pNiceMatrix}
%= Q_{m-1} \hathat{H}_{m-1}
\]
\[
\boxed{
    AQ_{m-1} = 
        \begin{pNiceMatrix}[vlines]
            Q_{m-1}\, \tilde{H}_{m-2}
            &
            Q_m \, {\bm h}_{m-1}
        \end{pNiceMatrix}
}
\]
\vspace{0.2em}

Then the last column of $Q_m = Q$, $\bm q_m$ only affects the last column of the result.

Define $\tilde{H}_{m-1} \in \RR^{m\times (m-1)}$ with same columns of $H$ except the last one, then we can write again as:
\[
\boxed{
    AQ_{m-1} = 
        Q_{m} \tilde{H}_{m-1}
}
\]

Now, let us generalize this result: we only keep the first $k$ columns from $AQ=QH$, and define $\overline H_k\in\RR^{m\times k}$ as the first $k$ columns of $H$. For $k\ll m$ most of the rows of $\overline H_k$ are zeros; $\tilde{H}_k\in\RR^{(k+1)\times k}$ is the same matrix after removing all lower rows we know are zero:
\[
    AQ = Q H \implies
        AQ_k = Q\overline H_k \implies
        AQ_k = Q_{k+1}\tilde{H}_k
\]

The Arnoldi iteration proposes something based on these observations, by calculating the elements of ``increasingly growing'' matrices.

In particular, the Arnoldi iteration generates, at each step/iteration $n$:
\begin{itemize}
    \item {}$Q_n\in\RR^{m\times n}$, with $n$ orthonormal vectors as columns ($Q_{n+1}$ contains the same columns of $Q_n$ and a new one).
    \item {}$\tilde H_n \in \RR^{(n+1)\times n}$, ``almost'' square and Hessenberg.
    
    $\tilde H_{n+1}$ adds a new row and column to $\tilde H_{n}$, but note that, for the new row, only the last entry is nonzero.
\end{itemize}

In any case, they must obey the property called ``Arnoldi relation'':
\[
    \boxed{
        A Q_k = Q_{k+1} \tilde H_k
    }\quad (k<m)
\]

Assuming that we know $Q_k$ and $\tilde H_{k-1}$, let us deduce the expression of the next vector $\bm q_{k+1}$.
To do so, expand the last column of both sides:
\[
    A\bm q_k = Q_{k+1} (\tilde{H}_k)_k
\]
where $(\tilde{H}_k)_k$ is the last column of $(\tilde{H}_k)_k$ (has $k+1$ entries).

Using $Q_{k+1}$ we write the RHS as linear combination of $\{\bm q_s\}_s$:
\[
    A\bm q_k = h_{1,k}\,\bm q_1 + h_{2,k}\,\bm q_2 + \cdots + h_{k,k}\,\bm q_{k} + h_{k+1,k}\,\bm q_{k+1} = \sum_{s=1}^{k+1} h_{s,k}\,\bm q_s
\]
Now, assuming $h_{k+1,k}\neq 0$, we can solve for $\bm q_{k+1}$:
\[
\implies
    \bm q_{k+1} = \frac{1}{h_{k+1,k}}
    \left(
        A\bm q_k - \sum_{s=1}^{k} h_{s,k}\,\bm q_s
    \right)
\]

So, to compute the next basis vector $\bm q_{k+1}$ we only need $A\bm q_{k}$ (MatVec with sparse $A$) and a linear combination of $k\ll m$ vectors of $m$ entries. Therefore, computing $\bm q_{k+1}$ requires $\OO(m)$ [TODO: check this].

But, for this, we first need to know the entries in the $k$-th column of $H$, $h_{s,k}$.

From the expression of $\bm q_{k+1}$, it is clear that $h_{k+1,k}$ is the normalizing factor that forces $\normtwo{\bm q_{k+1}}=1$; for the real case, there are two options (positive or negative), by convention we choose $h_{k+1,k}>0$.

We can be obtain the rest of $h_{s,k}$ from:
\[
    \bm q_i\transp (A \bm q_k) =
    \bm q_i\transp\left(\sum_{s=1}^{k+1} h_{s,k}\,\bm q_s\right)
    =
    \sum_{s=1}^{k+1} h_{s,k}\,\delta_{i,s}
    = h_{i,k}
    ,\quad \text{for } i\in \{1,..., k\}
\]
This requires $k$ scalar products (note that the MatVec $A \bm q_k$ is already calculated).


[In the algorithm in the book, $\bm q_{k+1}$ and all $h_{i,k}$ are calculated at the same time (also taking advantage of the orthogonal property), in order to save some computations ($h_{k+1,k}$ should be faster to obtain??)].

Also, whenever $\bm q_{k+1}=\bm 0$, we cannot continue building a base because any $A\bm q_{k}$ must be linearly dependent on the previous base $\{\bm q_i\}_{1 \le i\le k}$. This is not necessarily a bad outcome; what can be done it this situation depends on the application.

\subsection*{Arnoldi iteration and Krylov subspaces}

Now we try to apply the Arnoldi iteration in such a way that a basis of the Krylov subspace $\Kry_n$ is generated. Assume that $\dim(\Kry_n)=n$, then there exists an orthonormal basis $\{\bm q_i\}_{1\le i \le n}$ such that:
\[
    \Kry_n
        = \sspan(\bm b,\, A\bm b,\, A^2\bm b, \dots ,\, A^{n-1} \bm b)
        = \sspan(\bm q_1,\,\bm q_2,\,\bm q_3, \dots ,\,\bm q_n)
    \subseteq \RR^m
\]

To achieve this, set $\displaystyle\bm q_1 = \frac{\bm b}{\normtwo{\bm b}}$. Then, we have: $\Kry_1 = \sspan(\bm q_1)$.

And now, from the expression
\[    \bm q_{k+1} = \frac{1}{h_{k+1,k}}
    \Bigg(
        A\bm q_k - 
            \underbrace{
                \sum_{s=1}^{k} h_{s,k}\,\bm q_s
            }_{\in \sspan(\bm b, ..., A^{k-1}\bm b)}
    \Bigg)
\]

we can deduce (proven by induction?) that for $1<k<n$:
\[\text{$\bm q_{k}\in \Kry_k$, $\bm q_{k}\notin \Kry_{k-1}$ and $\bm q_{k+1}\notin \Kry_{k}$}\]

In conclusion: for $k \le n$, the sequence of $k$ vectors $\{\bm q_i\}_{1\le i\le k}$ produced by the Arnold iteration is a basis that generates the Krylov subspace $\Kry_k$ (for $k\le n\le m$).

This also means that, if the Arnoldi iteration finishes because it cannot find a new independent $\bm q_{k+1}$, then: $\Kry_k = \Kry_{k+1}$.

\subsection*{Arnoldi iteration and eigenvalues}

Consider again the Arnoldi relation:
\[
    AQ_n = Q_{n+1} \tilde H_{n}
\]

Premultiply both sides by $Q_n\transp$:
\[
    Q_n\transp AQ_n = Q_n\transp Q_{n+1} \tilde H_{n}
\]

And observe that:
\[
    Q_n\transp Q_{n+1} =
    \begin{pNiceMatrix}[vlines]
        I_n & \bm 0
    \end{pNiceMatrix} \in \RR^{n\times (n+1)}
\]

Then, we define $H_n$ as the $n\times n$ square matrix resulting from dropping the last row of $\tilde H_{n}$.

[Notation: $\overline{H}_n$ is a ``very tall'' matrix, $\tilde H_n$ is tall but ``almost square'' and $H_n$ is square; all of them have $n$ columns and contain the same nonzero elements (except that $H_n$ removes the lower right corner).]

Then, we can write:
\[
    \boxed{Q_n\transp AQ_n = H_{n}}
\]

Note that this is NOT a similarity transformation because $Q_n$ is not square (it is only \emph{semi}-orthogonal). But it would be so if we went go up to $n=m$ ($Q_m=Q$ is square).

And therefore, the eigenvalues of $H_m=H$ are the same as those in $A$.

This iterative method would be useful if the eigenvalues of $H_n$ (for $n\ll m$) somehow approximate those of $A$. This is, indeed, the case (proven in the next section) and we give them a name: the eigenvalues of $H_n$ are called ``Ritz values''.

We have that, in general, $\bm v\in\Kry_n$, then $A\bm v\notin\Kry_n$; but we can find the vector in $\Kry_n$ that is ``closest'' to $A\bm v$ by using projections (LS solution).

The projection matrix ``onto'' $\Kry_n$ is given by:
\[
    P_{\Kry_n} = Q_n Q_n\transp
\]
(recall that $\hat Q \hat Q\transp=P_{\range(\hat Q)}$ and 
$\hat Q\transp \hat Q=I$, for $\hat Q$ \emph{semi}-orthogonal)

Then we have
\[
    P_{\Kry_n} A = Q_n Q_n\transp A
\]
And also, applying change of basis on the left and right spaces, we obtain:
\[
    Q_n\transp (P_{\Kry_n} A) Q_n =
    Q_n\transp (Q_n Q_n\transp A) Q_n = 
    Q_n\transp A Q_n
\]
And, from the result above:
\[
    \boxed{Q_n\transp (P_{\Kry_n} A) Q_n = H_n}
\]

So $H_n$ is actually related to the ``projection'' of $A$ (whatever that means, because $A$ is not a vector...) onto the subspace $\Kry_n$.

To particularize this relation, we will use a polynomial approximation.

To begin with: any vector $\bm x\in\Kry_n$ (with $\dim\Kry_n=n$) can be uniquely written as
\[
    \bm x = c_0 \bm b +
    c_1 A\,\bm b +
    \cdots +
    c_{n-1} \, A^{n-1} \,\bm b
    =
    \sum_{i=0}^{n-1} c_i \, A^{i} \,\bm b
    =
    \left(
    \sum_{i=0}^{n-1} c_i \, A^{i}\right) \bm b
\]
this has the structure of an $(n-1)$-degree matrix polynomial \[q(Z)=\sum_{i=0}^{n-1} c_i Z^i\]
evaluated at $Z=A$ and then multiplied by $\bm b$.

That is:
\[
    \bm x = q(A)\,\bm b
\]

Let $\Pi^n$ be the set of $n$-th degree monic polynomials, that is, with the leading coefficient ($n$-th degree term) equal to 1. E.g.\@ the characteristic polynomial of any matrix.

Then, we look for a monic polynomial such that
\[
    \normtwo{p^{(n)}(A)\,\bm b}, \quad \text{for }p^{(n)}\in \Pi^n
\]
is minimized. This is an optimization problem on monic polynomials (the possible solutions are defined by lists of $n$ coefficients), which is called the ``Arnoldi/Lanczos Approximation Problem''.

And, actually, it can be proven (next section) that: if $\dim(\Kry_n)=n$ then the Arnoldi iteration produces the unique solution of the Arnoldi/Lanczos Approximation Problem, which happens to be the characteristic polynomial of $H_n$.

This motivates the following 2-phase eigenvalue algorithm:
\begin{enumerate}
    \item Perform the Arnoldi iteration (start with $\bm b$ random) up to step $n\ll m$.
    
    This results in the matrices $Q_n\in \RR^{n\times m}$, $H_n\in\RR^{n\times n}$.
    
    \item Compute the eigenvalues of $H_n$ (Ritz values).
    
    This may be done with a direct method because $n$ is small, but the QR iteration is also a valid option (especially using Givens rotations as many elements are already zero (??)).
\end{enumerate}

Now, we don't know which $n<m$ eigenvalues of $A$ are approximated by this algorithm and the ``quality'' of such approximations.

in practice the charpoly of hn has an approxation of the roots of , ie the eigenvalues.

\newpage
\subsection*{Proof of optimality in the Arnoldi Approximation Problem}

The Arnoldi Approximation Problem is as follows:

Given $A$ and $\bm b$.

Find the optimal monic polynomial $p^{(n)}\in \Pi^n$ such that
\[
    \normtwo{p^{(n)}(A)\bm b} \le \normtwo{q^{(n)}(A)\bm b},
    \qquad \forall  q^{(n)}\in \Pi^n
\]
Theorem: the solution of this problem is unique and is precisely the characteristic polynomial of $H_n$, $p^{(n)} = \chi_{H_n}$.

Proof:

Note that, for the case $m=n$, the characteristic polynomial $\chi_A=p^{(n)}$ is trivially a minimum, because: $\normtwo{\chi_A(A)\bm b}=\normtwo{0\bm b} = 0$ (we still need to prove that it is unique).

Any $q^{(n)}(A)\bm b$ with $q^{(n)}\in \Pi^n$ can be written as
\[
    q^{(n)}(A)\bm b = A^n \bm b + \bm z
    \quad \text{with }
    \bm z \in \Kry_{n}
\]

Now, because $\bm z\in \Kry_{n}$, we can write it as linear combination of $\{\bm q_i\}_{1\le i\le n}$:
\[ \bm z = -Q_n \bm y \text{ for some }\bm y \in \RR^n \]
(the reason for the minus sign becomes obvious later on)

Then, the minimization problem can be written as:
\[
    \min_{q^{(n)}\in \Pi^n} \normtwo{q^{(n)}(A)\,\bm b}=
    \min_{\bm y \in \RR^n} \normtwo{A^n\,\bm b - Q_n\,\bm y}
\]
This is (yet again) a Least Squares problem. Here, the residual is $\bm r=q^{(n)}(A)\bm b$, the system matrix is $Q_n$ and the RHS vector is $A^n\,\bm b$. That is, the system is:
\[
    Q_n\,\bm y = A^n\,\bm b
\]
Then, to obtain the LS solution, we orthogonally project $A^n\,\bm b$ onto $\range (Q_n)=\Kry_n$.

In particular, if the LS solution is $\bm y^*$, the orthogonal projection of $A^n\bm b$ must be $Q_n\bm y^*$. Then, the residual is minimized and therefore $q^{(n)}=p^{(n)}$ if $\bm y=\bm y^*$.

Now, because for $Q_n$ \emph{semi}-orthogonal we have $Q_n^{\dagger} = Q_n\transp$ and the LS solution is:
\[\bm y^* = Q_n\transp A^n \bm b\]

But this does not directly give us information about the optimal polynomial $p^{(n)}$.

Instead, let us deduce from the orthogonality condition:
\[
    \bm r\perp \Kry_n\implies
    Q_n\transp \bm r =\bm 0 \implies
    Q_n\transp ({A^n\,\bm b - Q_n\,\bm y^*}) =\bm 0
\]
And also (setting $q^{(n)}=p^{(n)}$ in the residual expression, as it produces the optimal solution):
\[
\bm r\perp \Kry_n\implies p^{(n)}(A)\bm b \perp \Kry_n
\]

Now, split the final similarity transformation of $A$ into submatrices:
\[
    A = Q HQ\transp
\]
\[
    Q = \begin{pNiceMatrix}
        Q_n & U
    \end{pNiceMatrix};\quad
    H = \begin{pNiceMatrix}
        H_n & X_1 \\
        X_2 & X_3
    \end{pNiceMatrix}
\]
Where we have the Arnoldi relation:
\[
    H_n = Q_n\transp A Q_n
\]
The columns of $Q_n$ span $\Kry_n$. 
Then, the columns of $U$ span $\Kry_n^\perp$ (orthogonal complement).

$X_1$ is a full matrix (\emph{fat}, as $n\ll m$), $X_2$ is all zeros except the upper right corner (\emph{tall}), $H_n$ is Hessenberg (square, $n\times n$, small), $X_3$ is Hessenberg (square, $(m-n)\times (m-n)$, big).

Now, following from the observation above:
\[
\bm r\perp \Kry_n\implies p^{(n)}(A)\bm b \perp \Kry_n
\implies Q_n\transp p^{(n)}(A)\bm b = \bm 0
\]

And, using the similarity trick:
\[
    A = QHQ\transp \implies p^{(n)}(A) = Q p^{(n)}(H) Q\transp
\]

Then, the condition can be written as:
\[
\bm r\perp \Kry_n\implies
    \underbrace{Q_n Q}_{(1)}
        p^{(n)}(H)
    \underbrace{Q\transp\bm b}_{(2)} = \bm 0
\]

For the left part $(1)$:
\[
    Q_n\transp Q = \begin{pNiceMatrix}[vlines]
        I_n & 0
    \end{pNiceMatrix}\in\RR^{n\times m}
\]
For the right part $(2)$, because $\bm q_1 = \bm b/\normtwo{\bm b}$:
\[
    Q\transp \bm b = 
    \normtwo{\bm b}\bm e_1
    \in\RR^{m\times 1}
\]
Then, a right multiplication by $Q\transp \bm b$ keeps the first column and a left multiplication by $Q_n\transp Q$ keeps the first $n$ rows.

Denote $(p^{(n)}(H))_{1:n,1}$ as the first $n$ entries in the first column of $p^{(n)}(H)$, then:
\[
\bm r\perp \Kry_n\implies
    \normtwo{\bm b}\, (p^{(n)}(H))_{1:n,1} = \bm 0
    \implies
    \boxed{(p^{(n)}(H))_{1:n,1} = \bm 0}
\]

To study the form of $(p^{(n)}(H))_{1:n,1}$, let us start analyzing $H^2$:
\[
    H^2
    =
    \begin{pNiceMatrix}
        H_n & X_1 \\
        X_2 & X_3 \\
    \end{pNiceMatrix}
    \begin{pNiceMatrix}
        H_n & X_1 \\
        X_2 & X_3 \\
    \end{pNiceMatrix}
    =
    \begin{pNiceMatrix}
        H_n^2 + X_1 \,X_2 & (\times) \\
        X_2\, H_n + X_3\, X_2 & (\times) \\
    \end{pNiceMatrix}
\]
(we don't compute the parts marked as $(\times)$)

Because of the structure of $X_2$, if we right-multiply $X_1$ with it, we select the first column of $X_1$, and placing it as the last column of the product, the rest are zeros:
\[
    X_1 \, X_2 = \begin{pNiceMatrix}[vlines]
        0_{n\times(n-1)} & (X_1)_{:,1}
    \end{pNiceMatrix}
\]
Now, the important detail here is that the submatrix $(H_n^2 + X_1 \, X_2)$ has as its first column the first column of $H_n^2$.

The lower left submatrix results: $X_2\, H_n + X_3\, X_2$. Where $X_2\,H_n$ selects the last row of $H_n$ (has nonzero entries in the last 2 positions) and places them in the first row; in $X_3 X_2$ the first column of $X_3$ (again, 2 nonzeros at the beginning) is copied to the last column; then:
\[
    X_2\, H_n + X_3\, X_2 =
    \begin{pNiceMatrix}
        &&&0& \times& \times \\
        &&&& 0& \times \\
        &&&& & 0\\
        0\\
    \end{pNiceMatrix}
\]

For subsequent powers, it can be proven that the submatrix at this position has each time one more nonzero diagonal (first nonzero diagonal is ``shifted'' to the left). Then, e.g.\@ $H_n^3$ is added to a matrix with only last 2 columns nonzero.

Therefore, it can be proven (induction?) that in $H^i$ for $1\le i\le n$, the first column of the submatrix at the position of $H_n$ is left untouched.

And, in particular, we have that: $(H^i)_{1:n,1}=(H_n^i)_{1:n,1}$ (for $1\le i\le n$).

As the polynomial is $p^{(n)}(H)$ is formed by powers $H^i$ up to $H^n$, all of this implies: \[
    (p^{(n)}(H))_{1:n,1} = \bm 0
    \implies
    (p^{(n)}(H_n))_{1:n,1}= \bm 0
\]

And now we have the minimizing condition:
\[
\bm r \perp \Kry_n
    \implies
    (p^{(n)}(H_n))_{1:n,1}= \bm 0
\]

And this condition is always met by the characteristic polynomial $\displaystyle p^{(n)}(z) = \chi_{H_n}(z)$

(because of the Cayley-Hamilton theorem).

This proves the existence of a solution.

\subsubsection*{Proof of uniqueness}

Assume (for contradiction) that there is another $n$-th order monic polynomial $\tilde p^{(n)}\neq p^{(n)}$ for which $\normtwo{\bm r}$ is minimized.

Then:
$p^{(n)}(A)\,\bm b\perp \Kry_n \text{ and } \tilde{p}^{(n)}(A)\,\bm b\perp \Kry_n$.

Define the difference polynomial $q^{(n-1)}$, which must be at least one degree lower:
\[
    q^{(n-1)} := p^{(n)}- \tilde{p}^{(n)} \in P_{n-1}
\]

Now $q^{(n-1)}$ must also have the property:
\[
    q^{(n-1)} (A)\,\bm b \perp \Kry_n
\]

As the columns of $Q_n$ span $\Kry_n$:
\[
    q^{(n-1)} (A)\,\bm b \perp \Kry_n
    \implies
    Q_n\transp (q^{(n-1)} (A))\,\bm b = \bm 0
\]

But because $q^{(n-1)}$ is of degree $\le n-1$, we must have that $(q^{(n-1)} (A))\,\bm b \in \Kry_n$.

We have shown that the vector $(q^{(n-1)} (A))\,\bm b$ is both orthogonal to $\Kry_n$ and in $\Kry_n$; so it must be identically zero:
\[
    q^{(n-1)}(A)\bm b = \bm 0
\]
But, by assumption, $\dim(\Kry_n)=n$ and $q^{(n-1)}(A)\bm b\in\Kry_n$ is a linear combination (using the coefficients of the polynomial) of the basis vectors formed by $A^{i}\bm b$ ($0\le i \le n-1$).

Then, the result can only be zero if the coefficients of the polynomial are all zero: therefore $q^{(n-1)} = 0$, which contradicts the assumption $q^{(n-1)} = p^{(n)}- \tilde{p}^{(n)} \neq 0$.

In conclusion, the solution $p^{(n)}\in \Pi_n$ must be unique.

\subsection*{GMRES}

The Generalized Minimal Residual Method (GMRES) is an equation-solving algorithm based on Arnoldi iteration and Krylov subspaces.

[Important note: we implicitly assume that the initial guess is $\bm x_0=\bm 0$ and thus $\bm r_0=\bm b$.

Otherwise, the generated Krylov subspaces in GMRES are not $\Kry_n=\Kry_n(A, \bm b)$, but $\Kry_n(A, \bm r_0)$; the generated solutions are of the form $\bm x_0 + \Kry_n(A, \bm r_0)$. This assumption makes the following analysis simpler.]

Consider the system of linear equations
\[
    A\bm x =\bm b
\]
with $A\in\RR^{m\times m}$ square.

The Least Squares problem (unconstrained minimal residual) has the form:
\[
    \min_{\bm x\in\RR^{m}}\normtwo{
        \underbrace{\bm b - A\bm x}_{\bm r}
    }
\]
The residual is $\bm r = \bm b - A\bm x$.

We want instead, to study the problem with solutions constrained to $\Kry_n\subseteq \RR^{m}$:
\[
    \min_{\bm x\in\Kry_n}\normtwo{
        \bm b - A\bm x
    }
    \implies
    \bm x_n = \arg\min_{\bm x\in\Kry_n}\normtwo{
        \bm b - A\bm x
    }
\]

If $\bm x_n = \bm x^*$ is the exact solution by chance and $A$ is invertible, then:
\[
    \bm x_n = \bm x^* =A^{-1}\bm b
\]
And the residual would be zero. This happens when $\bm b\in \Kry_{n+1} \cap \range(A)$ [TODO: is it $\Kry_{n+1}$ or $\Kry_{n}$??].

Again, $A^{-1}$ can be expressed as a matrix polynomial evaluated at $A$ and this motivates the use of polynomials to find/approximate the solution.

Now, define $\bm r_n$ as the residual for the optimal solution $\bm x_n\in\Kry_n$; the sequence $\{\normtwo{\bm r_n}\}_n$ is nonincreasing (not proven here, but makes sense intuitively).

Worst possible case: $\normtwo{\bm r_n}$ stays constant until the last step, where it drops sharply.

Now, recall the Arnoldi relation:
\[
    AQ_n = Q_{n+1} \tilde H_n
\]

Any vector $\bm x\in \Kry_n\subseteq\RR^m$ can be expressed as $\bm x = Q_n\bm y$, for some unique $\bm y\in \RR^n$, then we can rewrite the problem to solve as:
\[
    \min_{\bm x\in\Kry_n}\normtwo{
        \bm r
    }
    =
    \min_{\bm x\in\Kry_n}\normtwo{
        {A\bm x - \bm b}
    }
    =
    \min_{\bm y\in\RR^{n}}\normtwo{
        {AQ_n\bm y - \bm b}
    }
\]
Using the Arnoldi relation:
\[
    \min_{\bm x\in\Kry_n}\normtwo{
        \bm r
    }
    =
    \min_{\bm y\in\RR^{n}}\normtwo{
        {Q_{n+1}\tilde H_n\bm y - \bm b}
    }
\]
The matrices $Q_{n+1}$, $\tilde H_n$ are much smaller than $A$.

[TODO: $Q_{n+1}\transp$ is not even semi-orthogonal (it is the rows that are orthonormal, not the columns), but left multiplication preserves the 2-norm in this case; need proof of this...]

Then:
\[
    \min_{\bm x\in\Kry_n}\normtwo{
        \bm r
    }
    =
    \min_{\bm y\in\RR^{n}}\normtwo{
        Q_{n+1}\transp Q_{n+1}\tilde H_n\bm y - Q_{n+1}\transp\bm b
    }
    =
    \min_{\bm y\in\RR^{n}}\normtwo{
        \tilde H_{n} \bm y - Q_{n+1}\transp\bm b
    }
\]

Now, recall that $\bm q_1 =\frac{\bm b}{\normtwo{\bm b}}$, and therefore:
\[Q_{n+1}\transp\bm b = \normtwo{b} \bm e_1 \]
where $\bm e_1 \in \RR^{n+1}$ is the first canonical basis vector (first 1, rest zeros).

Then:
\[
    \min_{\bm x\in\Kry_n}\normtwo{
        \bm r
    }
    =
    \min_{\bm y\in\RR^{n}}\normtwo{
        \tilde H_{n} \bm y - \normtwo{\bm b}\bm e_1
    }
\]

The matrix $\tilde H_{n}$ is Hessenberg and ``almost''-square (tall, by one extra row) and we assume that $Q_{n+1}\transp\bm b$ is already computed (vector in $\RR^{n+1}$).

The problem is reduced to the ``almost''-square system ($n+1$ equations and $n$ unknowns):
\[
    \tilde H_{n} \bm y = \normtwo{\bm b} \bm e_1
\]
with the same residual as the original problem \[
    -\bm r_n = A\bm x_n - \bm b = \tilde H_{n} \bm y - \normtwo{\bm b} \bm e_1
\]
The GMRES algorithm to solve $A\bm x = \bm b$ is then:
\begin{enumerate}
    \item Set $\bm q_1 = \frac{\bm b}{\normtwo{b}}$.

    \item Execute the Arnoldi iteration until step $n$ to build the matrices $Q_{n+1}\in \RR^{m\times (n+1)}$ and $\tilde H_n \in \RR^{(n+1)\times n}$.
    
    \item Use a Least Squares method to solve the system
    \[
        \tilde H_{n} \bm y = \normtwo{\bm b} \bm e_1,
        \qquad  (\bm e_1 \in \RR^{n+1})
    \]
    obtaining the LS solution $\bm y_n \in \RR^{n}$.
    
    The QR method with Givens rotations is useful here, because $\tilde H_{n}$ is Hessenberg: we only need to zero out the first subdiagonal.
    
    \item Compute the final solution: $\bm x_n = Q_n\bm y_n$.
\end{enumerate}

\subsection*{``Standard'' GMRES}
The presented algorithm assumes that we know the number of iterations $n$ beforehand.

In practice, $n$ is selected according to a termination criterion, instead of using a fixed $n$.

To do so, we would compute at the $n$-th iteration the residual norm $
    \displaystyle\min_{\bm x\in\Kry_n}\normtwo{
        \bm r
    }$ and stop when it falls below a chosen tolerance.

Assuming that the method to compute the solution $\bm y_n$ is the QR method (e.g.\@ Givens rotations, as recommended); we have the decomposition:
\[
    \tilde{H}_n = \overline Q_n \, \overline R_n
\]
with $\overline Q_n$ square and orthogonal and $\overline R_n$ upper triangular, with the same size as $\tilde{H}_n$ (almost square).

The last row of $\overline R_n$ is all zeros, because it is upper triangular.

Then, we can rewrite again the residual norm:
\begin{align*}
% \[
    \displaystyle\min_{\bm x\in\Kry_n}\normtwo{
        \bm r
    }
    &=
    \min_{\bm y\in\RR^{n}}\normtwo{
        \tilde H_{n} \bm y - \normtwo{\bm b}\bm e_1
    }
    =
    \min_{\bm y\in\RR^{n}}\normtwo{
        \overline Q_n \, \overline R_n \bm y - \normtwo{\bm b}\bm e_1
    }=%\]
\\[0.5em]
%\[
    &=
    \min_{\bm y\in\RR^{n}}\normtwo{
        \overline R_n \bm y - \overline Q_n\transp \normtwo{\bm b}\bm e_1
    }
\end{align*}
%\]

Define $\overline{\bm q}_n=\overline Q_n\transp\bm e_1$ (last row of $\overline Q_n$); then:
\[
    \displaystyle\min_{\bm x\in\Kry_n}\normtwo{
        \bm r
    }
    =
    \min_{\bm y\in\RR^{n}}\normtwo{
        \overline R_n \bm y - \normtwo{\bm b} \overline{\bm q}_n
    }
\]

Now, observe the structure:
\[
\overline R_n \bm y - \normtwo{\bm b} \overline{\bm q}_n
=
\begin{pNiceMatrix}
    \times & \times & \times & \cdots & \times
\\
           & \times & \times & \cdots & \times
\\
           &        & \times & \cdots & \times
\\
           &        &        & \ddots & \vdots
\\
           &        &        &        & \times
\\\hline
&&\bm 0
\end{pNiceMatrix}
\begin{pNiceMatrix}
    \times
\\
    \times
\\
    \times
\\
    \vdots
\\
    \times
\end{pNiceMatrix}
-
\begin{pNiceMatrix}
    \times
\\
    \times
\\
    \times
\\
    \vdots
\\
    \times
\\\hline
    \alpha^{(n)}
\end{pNiceMatrix}
=
\begin{pNiceMatrix}
    \times
\\
    \times
\\
    \times
\\
    \vdots
\\
    \times
\\\hline
    0
\end{pNiceMatrix}
-
\begin{pNiceMatrix}
    \times
\\
    \times
\\
    \times
\\
    \vdots
\\
    \times
\\\hline
    \alpha^{(n)}
\end{pNiceMatrix}
\overset{*}{=}
\begin{pNiceMatrix}
    0
\\
    0
\\
    0
\\
    \vdots
\\
    0
\\\hline
    -\alpha^{(n)}
\end{pNiceMatrix}
\]
Assuming $\tilde H_n$ is full rank, the last equality ($\overset{*}{=}$) can always be attained (otherwise this may or may not be possible). The last element $\alpha^{(n)}$ is always the same regardless of $\bm y$. Note that this assumption means that all the diagonal elements of $R$ are nonzero.

This result implies that the minimum norm residual of the system
$\overline R_n \bm y = \normtwo{\bm b} \overline{\bm q}_n$
(with same norm as in the system $A\bm x = \bm b$) is all zeros except for the last entry.

Then:
\[
\min_{\bm x\in\Kry_n}\normtwo{
        \bm r
    }
    = |\alpha^{(n)}| =
    \normtwo{b}\, |\overline q_{nn}|
\]
Where $\overline q_{nn}$ is the entry at the lower left corner of $\overline{Q}_n$.

Then, the ``standard'' GMRES algorithm is as follows:
\begin{enumerate}
    \item Set $\bm q_1 = \frac{\bm b}{\normtwo{b}}$, $n=1$.
    
    \item Execute one step of the Arnoldi iteration to build the matrices $Q_{n+1}\in \RR^{m\times (n+1)}$ and $\tilde H_n \in \RR^{n\times (n+1)}$.
    
    \item Perform the QR decomposition (recommended to use Givens to remove just one subdiagonal) \[\tilde{H}_n = \overline Q_n \, \overline R_n.\]
    
    (For the next step we only need $\overline{q}_{nn}$, but it requires the whole QR decomposition... [??])
    
    \item Compute the value:
    \[
        \min_{\bm x\in\Kry_n}\normtwo{
            \bm r
        }
        = |\alpha^{(n)}| =
        \normtwo{\bm b}\, |\overline { q}_{nn}|
    \]
    (with $\overline { q}_{nn}$ the element at the lower left corner of $\overline{Q}_n$)
    
    Test if $|\alpha^{(n)}|$ is below a given threshold/tolerance:
    
    \begin{itemize}
        \item If not, set $n:=n+1$ and goto step 2.
        \item Otherwise: we finished iterating, continue in step 5.
    \end{itemize}
    
    \item Find the Least Squares solution $\bm y_n$ of the system
    \[
        \tilde H_{n} \bm y = \normtwo{\bm b} \bm e_1,
        \qquad  (\bm e_1 \in \RR^{n+1})
    \]
    
    Using the computed QR decomposition: $\tilde{H}_n = \overline Q_n \, \overline R_n$.
    \[
    \overline Q_n \, \overline R_n \bm y_n = \normtwo{\bm b} \bm e_1
    \implies 
    \overline R_n \bm y_n = \normtwo{\bm b}\, \overline Q_n\transp \bm e_1 = \normtwo{\bm b}\hat{\bm w} = \bm w
    \]
    
    Where be define $\bm w=\overline R_n \bm y_n$; $\hat{\bm w}$ is its normalized vector and is the first row of $\overline Q_n$.
    
    That is, the steps to perform are:
    \[ \hat{\bm w} := (\overline Q_n)_{1:} \text{ (first row of $\overline Q_n$)} \]
    \[ {\bm w} := \hat{\bm w} \,\normtwo{b} \]
    \[ \text{Solve for } {\bm y}_n \text{ in }
        \overline R_n \bm y_n = \bm w
        \quad \text{(use back substitution)}
    \]

    \item Compute the final solution: $\bm x_n = Q_n\bm y_n$.

\end{enumerate}

\subsection*{Convergence of GMRES}

This method is ``hard'' in general to analyze, because of nonlinearity.

But for some ``interesting'' matrices the analysis becomes easier/feasible; from harder to easier, we have: general matrices (possibly defective), diagonalizable/non-defective matrices and normal matrices (including Hermitian/real symmetric ones).

Assuming the initial guess $\bm x_0=\bm 0$, then the initial residual is $\bm r_0=\bm b$.

GMRES solves iteratively the problems (for $n$):
\[
\normtwo{
    \bm r_n
}=
\min_{\bm x\in \Kry_n}\normtwo{
    \bm r
}=
\min_{\bm x\in \Kry_n}\normtwo{
    \bm b - A\bm x
}
\]
A vector $\bm x_n\in \Kry_n$ can be expressed as:
\[
    \bm x_n = \sum_{s=0}^{n-1}
        \gamma_s A^s \,\bm b
        =\sum_{s=0}^{n-1} \gamma_s A^s \bm r_0
\]

Then, the $n$-th residual is in $\Kry_{n+1}$:
\begin{align*}
     \bm r_n &= \bm b - A\bm x_n
    =\bm r_0 - A\bm x_n
    =\bm r_0 - A\left(
        \sum_{s=0}^{n-1} \gamma_{s+1} A^s \bm r_0
    \right)=
    \bm r_0 - \left(
        \sum_{s=1}^{n} \gamma_{s} A^s \bm r_0
    \right)=
    \\
    &=
     \left(I - 
        \sum_{s=1}^{n} \gamma_{s} A^s
    \right) \bm r_0
    =\overline{p}^{(n)}(A)\bm r_0; \quad
    \text{with }
    \overline{p}^{(n)} \in \mathrm{P}_n
\end{align*}

where $\rP_n$ is the set of polynomials $\overline{p}$ of degree $n$ with $\overline{p}(0)=1$.

\subsection*{Results for general matrices}

Theorem: let $A$ be nonsingular, and $\bm x_n$ the $n$-th GMRES iteration solution. Then, for all $\overline{p}^{(n)}\in\rP_n$, there exists an optimum $\overline{p}_*^{(n)}\in\rP_n$ such that
\[
    \normtwo{\bm r_n} =
        \normtwo{\overline{p}_*^{(n)}(A)\bm r_0}=
        \min_{\overline{p}(A)\in\rP_n}\normtwo{\overline{p}(A)\bm r_0}
        \le \normtwo{\overline{p}^{(n)}(A)\bm r_0}
\]
Proof: [Omitted]

A consequence of this is (from the definition of the matrix 2-norm):
\[
\normtwo{\bm r_n}
    \le
    \normtwo{\overline{p}^{(n)}(A)\bm r_0}
    \le
    \normtwo{\overline{p}^{(n)}(A)}\normtwo{\bm r_0}
\]

Then:
\[
\boxed{
    \frac{\normtwo{\bm r_n}}{\normtwo{\bm r_0}} \le
        \normtwo{\overline{p}^{(n)}(A)}
    ,\quad
    \forall \overline{p}^{(n)}\in\rP_n
}
\]

The expression ${\normtwo{\bm r_n}}/{\normtwo{\bm r_0}}$ is called \emph{relative residual}, and is the standard metric used to assess the performance of GMRES.

Theorem: for $A$ nonsingular, GMRES finds the solution within $m$ iterations.

Proof: use the characteristic polynomial $\chi_A(z)=\det(zI-A)$, but note that, $\chi_A$ is not guaranteed to be in $\rP_n$  because $\chi_A(0)=\det(-A)$.

But $\det(A)\neq 0$ and the optimal polynomial $p_*^{(m)}$ can be simply constructed as 
\[
    \boxed{p_*^{(m)} = \frac{\chi_A}{\det(-A)} \in \rP_n}
\]

Then:
\[
    \normtwo{\bm r_m} \le
    \normtwolr{\frac{\chi_A(A)}{\det(-A)}\,\bm r_0} = 0
    \implies 
    \bm r_m = \bm 0
\]

\subsection*{Results for diagonalizable matrices}

For $A$ diagonalizable we have:
\[
    A = V \Lambda V^{-1}
\]
In that case, the columns of $V$ are eigenvectors and $\Lambda$ diagonal contains the associated eigenvalues in the same order.

Recall that, for any polynomial $p$ we can write:
\[
    p(A) = V p(\Lambda)\, V^{-1}
\]

Theorem: let $A = V \Lambda\, V^{-1}$ be diagonalizable and nonsingular, then for all $\overline p\in \rP_n$,
\[
\frac{\normtwo{\bm r_n}}{\normtwo{\bm r_0}} \le
    \kappa_2(V)\,
    \max_{z\in\delta(A)}
    |\overline{p}(z)|
\]
Where $\kappa_2(V)=\normtwo{V}\normtwo{V^{-1}}$ is the condition number of $V$ and $\delta(A)$ is the set of eigenvalues of $A$.

Proof:

From the general theorem:
\[
\frac{\normtwo{\bm r_n}}{\normtwo{\bm r_0}} \le
    \normtwo{\overline p(A)} \le
    \normtwo{V}\normtwo{V^{-1}}
    \normtwo{\overline p^{(n)}(\Lambda)}
\]

And as ${p^{(n)}(\Lambda)}$ is diagonal $\normtwo{p^{(n)}(\Lambda)}$ is the absolute maximum element in it (abs.\@ eigenvalues of $\normtwo{p^{(n)} (A)}$)

Note: recall that:
\[
    \normtwo{A} = \sqrt{\rho(A\transp A)} = \sqrt{\rho(A A\transp)}
\]


\subsection*{Results for normal matrices}

\end{document}
